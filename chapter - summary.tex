\chapter{Summary}
In conclusion, there's a lot of parallel between both coherent transient phenomena of optical forerunners/precursors and superflash in terms of their formulation. This is especially true from the viewpoint of how researchers are studying the behaviour of both phenomena. As we can see from all of the researches and papers that have been reviewed, both the precursor and superflash researches employ the same FFT/IFFT approach in numerically simulating their transmitted signal. From this FFT/IFFT approach, various insights have been gained both from the precursor and coherent flash/superflash experiments that have been discussed extensively in the previous sections. As an example, in \cite{Macke2013}, using a combination of FFT and saddle-points method, they succeeded in deriving an analytical expression for both the Sommerfeld and Brillouin forerunner that is also equally valid even when these forerunners overlaps each other to form precursors (when in optically thin medium).\\

On the other hand, one of the instance where FFT/IFFT might have been applied in simulating the transmitted signal for coherent flash/superflash experiment can be seen in \cite{Kwong2014}. There, they had mentioned about numerically performing an inverse Fourier transform to the spectrum of transmitted field $E(\omega, z)$ to acquire the time-evolution of the transmitted field $E(t, z)$. From this ability to numerically calculate the inverse Fourier transform of an output signal in the frequency domain (e.g. using FFT/IFFT framework), they succeeded in observing the phenomena of amplified coherent flash (known as superflash) and also extracted some criteria (e.g. optical thickness and incident field detuning) on how it can be achieved. So, based on the various research papers that had been reviewed, most of them incorporated the FFT/IFFT approach for their numerical simulation either on its own or alongside other numerical method such as the numerical asymptotic method that is frequently seen in optical precursor research papers. Therefore, it is evident that FFT/IFFT approach are really relevant and reliable in acquiring the results desired with quick and efficiently with low computing cost.\\

So, through this final year project, the main focus will be on developing a numerical simulation program in the widely-known Python programming language that will implement FFT/IFFT to simulate the temporal profile of the transmitted intensity $I_{t}(t)$ and the (detuning) spectral profile of the forward-scattered intensity $I_{s}(\delta)$. More specifically, from this general FFT/IFFT simulation program, the project will focus on studying/comparing the difference of the superflash phenomena in 2 different medium that is the 2-level medium (has been conducted in \cite{}) and 3-level EIT-enabled medium. Based on what we have collected from previous literature reviews, the study of coherent transient phenomena specifically the coherent superflash in 3-level EIT medium has not been done yet by any group. From this project, it is hoped that it can provide some new insights or extensions of the current knowledge on the behaviour of superflash in a new type of medium (3-level EIT-enabled medium).
