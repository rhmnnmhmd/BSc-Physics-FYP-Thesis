\chapter{Summary}
In conclusion, there's a lot of parallel between both coherent transient phenomena of optical forerunners/precursors and superflash in terms of their formulation. This is especially true from the viewpoint of how researchers are studying the behaviour of both phenomena. From the literature review, both the precursor and superflash researches employ the same FFT/IFFT approach in numerically simulating their transmitted signal. From this FFT/IFFT approach, various insights have been gained both from the precursor and coherent flash/superflash experiments that have been discussed extensively in the previous sections. As an example, in \cite{Macke2013}, using a combination of FFT and saddle-points method, they succeeded in deriving an analytical expression for both the Sommerfeld and Brillouin forerunner that is also equally valid even when these forerunners overlaps each other to form precursors (when in optically thin medium).

On the other hand, one of the instance where FFT/IFFT might have been applied in simulating the transmitted signal for coherent flash/superflash experiment can be seen in \cite{Kwong2014}. There, they had mentioned about numerically performing an inverse Fourier transform to the spectrum of transmitted field $E(\omega, z)$ to acquire the time-evolution of the transmitted field $E(t, z)$. From this ability to numerically calculate the inverse Fourier transform of an output signal in the frequency domain (e.g. using FFT/IFFT framework), they succeeded in observing the phenomena of amplified coherent flash (known as superflash) and also extracted criteria (e.g. optical depth $OD$ and incident probe laser detuning $\Delta$) on how it can be achieved. So, based on the various research papers reviewed, most of them incorporated the FFT/IFFT approach for their numerical simulation either on its own or alongside other numerical method such as the numerical asymptotic method that is frequently seen in optical precursor research papers. Therefore, it is evident that FFT/IFFT approach are really relevant and reliable in acquiring the results desired quickly and efficiently with low computing cost.

So, through this final year project, a FFT/IFFT-related numerical simulation was developed in Python programming language to simulate the temporal profile of the transmitted intensity $I_{t}(t, z)$ and the detuning spectrum of the forward-scattered intensity (known as the $I_{s}$ spectrum throughout this project). The behaviour of the superflash was studied extensively in this project by comparing its behaviour in the conventional 2-level medium and in 3-level EIT medium. This project specifically focused on the relation between the $I_{s}$ spectrum and the phase shift spectrum $\Delta\phi$. Based on the literature review, the study of coherent transient phenomena specifically the probe-detuned coherent superflash in 3-level EIT medium has not been investigated by any group therefore I hope this project can provide new insights or extensions of the current knowledge on the behaviour of superflash in a new type of medium (3-level EIT medium). 
