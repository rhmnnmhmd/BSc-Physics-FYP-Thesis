%!TEX root=thesis.tex
%This is the draft abstract. More soon, I promise!
%Ini merupakan abstrak dalam Bahasa Melayu (satu perenggan).

Superflash koheren adalah salah satu fenomena transien koheren yang dapat diperhatikan apabila pancaran laser probe yang terjadi pada media secara tiba-tiba dimatikan sama ada melalui amplitud atau modulasi fasa isyarat kejadian. Untuk mempelajari lebih lanjut mengenai tingkah laku superflash koheren, simulasi berangka dari isyarat yang dihantar telah dirancang dalam bahasa pengaturcaraan Python menggunakan paket Fast Fourier Transform (FFT) untuk mendapatkan evolusi masa dari isyarat yang dihantar $ E (t, z $ dengan mengira Inverse Fast Fourier Transform (IFFT) spektrum isyarat yang dihantar $ E (\ omega, z) $. Objektif projek ini adalah untuk menambah pengetahuan mengenai superflash koheren mengenai parameter ciri seperti keamatan yang tersebar ke hadapan $ I_ {s} $ kepada jenis medium baru yang lain (medium EIT 3 tingkat) dan bukannya medium 2 peringkat biasa seperti yang dilakukan di \ cite {Kwong2014}. Hasil penting yang dijumpai dari projek ini adalah bagaimana spektrum pelepasan medan yang tersebar ke hadapan untuk medium EIT 3 tingkat yang dihasilkan di sini berbeza dengan Rajah 1) d) yang dihasilkan dalam \ cite {Kwong2014}. Secara ringkasnya, dapat dinyatakan bahawa hasil yang dihasilkan di sini menunjukkan bahawa ketika medium diubah dari medium 2-level menjadi medium EIT 3-level, fluktuasi akan mulai ada di wilayah sekitar 0 yang tidak bertentangan dengan level 2 versi di mana terdapat sedikit turun naik. Sebagai kesimpulan, kod simulasi berangka dari projek ini akan berguna kepada pelajar atau bahkan penyelidik untuk mengkaji tingkah laku temporal intensiti yang dipancarkan dan tingkah laku spektrum keamatan yang tersebar ke hadapan.

\textbf{Kata kunci:} Superflash, Medium EIT 3-Aras , Intensiti Serakan-Hadapan.
