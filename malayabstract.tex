%!TEX root=thesis.tex
%This is the draft abstract. More soon, I promise!
%Ini merupakan abstrak dalam Bahasa Melayu (satu perenggan).

Superflash koheren adalah salah satu fenomena transien koheren yang dapat diperhatikan apabila pancaran laser probe insiden yang menyebarkan melalui media dimatikan secara tiba-tiba sama ada melalui amplitud atau modulasi fasa. Untuk mempelajari lebih lanjut mengenai tingkah laku superflash koheren, simulasi berangka isyarat yang dihantar telah dibuat dalam bahasa pengaturcaraan Python. Bahasa ini menggunakan pakej Fast Fourier Transform (FFT) untuk mendapatkan evolusi masa dari isyarat yang dihantar $ E_ {t} (t, z) $ dengan mengira Inverse Fast Fourier Transform (IFFT) dari spektrum isyarat yang dihantar $ E_ { t} (\ omega, z) $. Objektif projek ini adalah untuk menambah pengetahuan mengenai superflash koheren mengenai parameter ciri seperti keamatan yang tersebar ke hadapan $ I_ {s} $ kepada jenis medium baru yang lain iaitu medium Ketelusan Induksi Elektromagnetik 3-tahap (EIT) dan bukannya medium 2-tahap biasa seperti yang dibincangkan dalam \ cite {Kwong2014} Hasil penting yang dijumpai dari projek ini adalah ketetapan detas dari superflash yang tidak dapat diperoleh dalam medium 2 tingkat. Secara ringkasnya, dapat dinyatakan bahawa hasil yang dihasilkan di sini menunjukkan bahawa apabila medium diubah dari medium 2-level menjadi medium EIT 3-level, ciri spektrum unik diperhatikan dekat resonansi medium yang bertentangan dengan kes 2 level di mana ia monoton tanpa ciri menarik.

%\textbf{Kata kunci:} Superflash, Medium EIT 3-Aras , Intensiti Serakan-Hadapan.
