%!TEX root=thesis.tex
%This is the draft abstract. More soon, I promise!
%Ini merupakan abstrak dalam Bahasa Melayu (satu perenggan).

Superflash (koheren) adalah salah satu fenomena transien koheren yang dapat diperhatikan apabila pancaran prob laser yang melalui sesuatu medium dimatikan secara tiba-tiba sama ada melalui modulasi amplitud atau modulasi fasa. Untuk mengkaji lebih lanjut mengenai tingkah laku superflash, sebuah simulasi numerikal dalam \textit{Python} (salah satu bahasa pengaturcaraan) telah dihasilkan untuk mensimulasikan medan elektrik yang tersebar. Kami menggunakan pakej Fast Fourier Transform (FFT) di dalam \textit{Python} untuk mendapatkan evolusi-masa bagi medan elektrik yang tersebar $E_{t}(t, z)$. Ini dilakukan dengan mengira Inverse Fast Fourier Transform (IFFT) untuk spektrum kepunyaan medan elektrik yang tersebar $E_{t}(\omega, z)$. Objektif projek ini adalah untuk menambah pengetahuan mengenai salah satu ciri-ciri superflash iaitu intensiti serakan-hadapan $I_{s}$ dalam medium yang baru. Medium yang digunakan dalam projek ini ialah  dan bukannya medium 2-tahap seperti yang dibincangkan dalam \cite{Kwong2014}. Hasil penting yang dijumpai daripada projek ini adalah kebolehan untuk mengawal superflash. Ini tidak boleh dilakukan dalam medium 2-tahap. Secara ringkasnya, apabila medium ditukar daripada medium 2-tahap kepada medium Ketelusan Induksi Elektromagnetik (EIT) 3-tahap, terdapat ciri-ciri spektrum yang menarik terbentuk berdekatan dengan titik resonans yang mana ciri-ciri ini tidak wujud dalam medium 2-tahap.

\textbf{Kata kunci:} Superflash, Medium Ketelusan Induksi Elektromagnetik (EIT) 3-tahap, Intensiti Serakan-Hadapan, Laser Gandingan, Frekuensi Rabi.
