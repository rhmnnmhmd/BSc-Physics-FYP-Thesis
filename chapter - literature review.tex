\chapter{Literature Review}
As an overview of what will be mostly covered in this literature review, some of the recently studied phenomena related to coherent light-matter interaction will be explained briefly in this section. Because this review will focus more on transient behaviour of light and how this transient component of transmitted light interacts with the main signal, we first need to understand one of the earliest ideas/studies. For example, one of the original studies regarding the unique behaviour of transient light is in the paper by Segard, Zemmouri and Macke \cite{Segard_1987} where they succeeded in generating transmitted electromagnetic pulse that have amplitude that is significantly larger (up to 3 times enhanced) than its input amplitude. At that time this result was very counter-intuitive because we would usually expect that a light that is transmitted through a medium will have its output amplitude bounded by its input amplitude whereas this wasn't the case in their experiment.\\

In addition to that, in this literature review, we also learn more on this idea of "amplified" transmitted light by reviewing other recent papers that utilize this concept in their experimental research that have succeeded in associating this behaviour of amplified transient light with other existing concept. As an example, in \cite{jeong2010slow}, they managed to generate coherent stacked optical transients and identify it as optical precursors. These kinds of paper are really interesting because we could see how certain unique physical behaviour of light that was observed in an experimental setup actually are related to a theoretical concept that was devised years before then.\\

Last but not least, we will also look at some researches that had published about other kinds of observation of optical precursors such as in \cite{Du2008} where they had observed optical precursors at the scale of bi-photon. They could identify that what they had observed is indeed an optical precursor by utilizing a widely known approximation method (stationary-phase approximation) where they could conclude that the spike that they observed in their plot is a result of interference between high- and low- frequency spectrum in the optical precursor itself. Then, we will also look at some recent development that have been done on optical precursor theory for example in \cite{Chen2010} where they have discovered that 2 optical transient phenomena that have been regarded as different phenomena, free-induction decay and optical precursor are actually the same physical process  but different realizations of the process in different conditions.\\

In summary, the goal of this literature review is to review and compile some of the researches on coherent transient light-matter interaction e.g. free-induction decay (FID)(has been successfully identified as optical precursor in \cite{Chen2010}), and especially superflash phenomena \cite{Kwong2014}. We'll review it systematically starting from its inception up to the most recent studies. We'll also discuss some details in these researches from their experimental methods used in analyzing their data; their results; and future advancement and development proposed by each researcher for example on how these knowledge of coherent transients can be successfully applied to real applications: medicine, manufacturing industry, etc.\\

We finally focus on the main idea of this literature review that is about coherent superflash phenomena. This phenomena is quite similar to the optical precursor phenomena in the fact that both are ultra-fast phenomena lasting only for a short amount of time. In addition to that, the methods used in recent/most papers (on coherent flash or superflash) in comparing between experimental and numerical results are FFT/IFFT similar to the ones used in the various papers that have been cited in the previous sections. So, coherent transient phenomena (super-flash) will be the main focus of this project transitioning from the idea of precursors/forerunners that was introduced in previous sections. We'll review some of the recent papers that have been published in the previous 5 to 7 years regarding the phenomena of coherent flash of light, more specifically the super-flash effect.\\

One of the earlier paper that studied this coherent flash phenomena is in \cite{Chalony2011}. In this paper, they sent a laser light that is on-resonance with the target medium and discovered a coherent flash (almost 100\% of incident signal) that occurred when the incident laser light is suddenly switched on/off, basically having a step-on or step-off modulation as used in optical precursor studies. This was really surprising because typically when a coherent light beam (laser) are shined on a medium, most of it will be absorbed in the medium and only some of it will be transmitted through the material with transmitted intensity obeying Beer-Lambert's law;

\begin{align}
    I_{t}(z) &= I_{0} e^{-a_{0} z}\\
    OD &= a_{0} z
\end{align}

Where;

\begin{itemize}
    \item $I_{t}(z)$ is the transmitted intensity
    \item $I_{0}$ is the incident/initial intensity
    \item $OD$ is the optical depth
    \item $a_{0}$ is resonant absorption coefficient
    \item $L$ medium thickness
\end{itemize}

For their experiment, they managed to observe the coherent flash signal by utilizing the FID mechanism that is the emission of coherent light after the incident signal is abruptly switched off. Basically, what they had done is to decrease the amount of destructive interference between incident field $E_0$ and forward scattered field $E_s$ to achieve increased intensity in the transmitted field $E_t$. One important thing is that the medium used must be optically thick to produce significant coherent flash because this coherent flash comes from the collective radiation of the dipoles in the medium that doesn't instantaneously decay when the incident field are switched off.\\

Even then, one of the main problem encountered by researchers in the observation of coherent flash is the extremely short time scale at which it exists (in the range of fsec or psec). Fortunately, it is found out that this problem can be overcome by using medium with resonant scatterers with longer lifetime. Consequently, the phenomena of coherent flash was first observed and reported in a Nuclear Magnetic Resonance (NMR) experiment in \cite{Hahn1950} while in the optical domain, it was first observed in \cite{Brewer1972, Foster1974}. In \cite{Chalony2011}, they focused their study on FID that is causing the coherent flash in an optically thick medium and finding out how the coherent flash (its characteristic decay time) vary with optical thickness and temperature. The three main highlights from the results in \cite{Chalony2011} is;

\begin{itemize}
    \item The coherent flash decay time $\tau$ is inversely proportional to the rms atomic velocity $\Bar{\nu}$. Because the rms atomic velocity $\bar{v}$ is directly proportional to the temperature, then the coherent flash decay time are inversely proportional to the temperature $T$
    \item The coherent flash decay time $\tau$ is inversely proportional with the optical depth $OD$
    \item The relative phase $\phi$ between input field $E_{0}(t, z)$ and transmitted field $E_{t}(t, z)$ are very small at a time scale that is smaller than the coherent flash initial decay time. But, at longer time scale, the relative phase value increase significantly until it reaches a constant value. Medium with higher OD have a larger value for the aforementioned constant value of the relative phase
\end{itemize}

Later, in the next few years after the paper \cite{Chalony2011} has been published, a new study on the super-flash phenomena as a special case \cite{Kwong2014}. This later paper focused more on the amplitude of the transmitted field instead of its decay time. They also shined laser light on an optically thick cold strontium atomic gas like in \cite{Chalony2011}. They observed a transmitted field with peak intensity more than 3 times the incident intensity. They concluded that this observed phenomena is possible because of the cooperative forward emission of the atoms in the target medium. The main difference between the observation of coherent flash/superflash in \cite{Kwong2014} and \cite{Chalony2011} is that the former introduced incident probe laser detuning $\Delta$ while the later didn't. Due to this detuning from medium resonance, they discovered that at certain detuning ($\abs{\delta} = 11.2 \Gamma$ in their experiment), they managed to get a transmitted signal with peak intensity about 3 times higher than incident intensity as in Figure \ref{fig: superflash}. The peak intensity for this super-flash is related to the FID mechanism involved in coherent flash. Basically, in coherent flash phenomena, the transmitted field $E_{t}$ are expressed as;

\begin{equation}
    E_{t} = E_{0} + E_{s}
\end{equation}

\begin{figure}[h!]
    \centering
    \includegraphics[scale = 1]{superflash}
    \caption{Plot of normalized transmitted intensity $\frac{I_{t}(t)}{I_{0}}$ against time for detuning $\delta = 11.2\Gamma$. Red curve represents incident square pulse intensity. The black curve represents experimental data. The blue line is the level for $\frac{I_t}{I_0}$. The green open circle represents the value of $\frac{I_{s}}{I_{0}}$. Figure courtesy of \cite{Kwong2014}.}
    \label{fig: superflash}
\end{figure}

Where $E_{t}$, $E_{0}$ and $E_{s}$ is the transmitted field, incident field and forward-scattered field respectively. For an on-resonance incident field, immediately before abrupt probe laser switch-off ($t = 0^{-}$), $E_{t} \approx 0$. So, $E_{s} = - E_{0}$. Then, immediately after probe switch-off, $E_{0} = 0$. So now, $E_{t} = E_{s}$. Therefore, combining this 2 resulting expression, we will get that $E_{t} = - E_{0}$ therefore giving us $I_{t} = I_{s}$. So, in the transient region immediately before and after probe switch-off, the transmitted intensity is directly related to the forward scattered intensity and in this case, equal to the forward scattered intensity. Therefore, this direct relationship between $I_{t}$ and $I_{s}$ in the region before/after abrupt probe switch-off is what allows for the transmitted intensity in Figure \ref{fig: superflash} to exceed the incident intensity when the incident field is detuned by a certain amount as shown in Figure \ref{fig: detuned}. 

\begin{figure}[h!]
    \centering
    \includegraphics[scale = 1]{detuned}
    \caption{Plot of normalized forward-scattered intensity against probe laser detuning. The green solid dots are experimental data. The black solid line is the theoretical prediction. Figure courtesy of \cite{Kwong2014}.}
    \label{fig: detuned}
\end{figure}

In addition to this, by conservation of energy, they showed that the forward-scattered intensity $I_{s}$ is limited by 4 times the incident intensity $I_{0}$. It is simply from the fact that;

\begin{equation}
    E_{t} = E_{0} + E_{s} \leq E_{0}
\end{equation}

Taking the modulus-squared of both side of the equation;

\begin{equation}
    \abs{E_{0} + E_{s}}^{2} \leq \abs{E_{0}}^{2}
\end{equation}

They interpreted this as the allowed value of $E_{s}$. The allowed value of $E_{s}$ will be in a circle of radius $\abs{E_{0}}$ centered at $-E_{0}$. Therefore, the maximum value of $E_{s}$ is $E_{s} = -2 E_{0}$. So, taking the square of both sides for intensity they finally get upper bound for the forward-scattered intensity, $I_{s} \leq 4 I_{0}$. To show this, they plotted a colored plot representing the relation of the normalized forward-scattered intensity $\frac{I_{s}}{I_{0}}$ with 2 parameters, the detuning $\delta$ and the optical thickness (at resonance) $b_{0}$ as shown in Figure \ref{fig: I_s/I_0}. As we can clearly see from Figure \ref{fig: I_s/I_0}, when the white line that represents super-flash occurrence are extrapolated, it will eventually reach the theoretical upper bound $\frac{I_{s}}{I_{0}} = 4$.

\begin{figure}[h!]
    \centering
    \includegraphics[scale = 1]{IsI0}
    \caption{Colored plot of $\frac{I_{s}}{I_{0}}$  against resonant optical thickness (non-zero temperature) $b_{\bar{v}}(\delta) = b_{\Bar{v}}(0)$ and detuning $\frac{\abs{\delta}}{\Gamma}$ for the temperature $T = 3.3(2) \mu K$. Black dashed line represents optical thickness of their experiment. The white solid line represents linear dependence of the detuning $\delta$ to the resonant optical thickness $b_{\bar{v}}(0)$ for when maximum value of $\frac{I_{s}}{I_{0}}$ occurs (super-flash). Figure courtesy of \cite{Kwong2014}.}
    \label{fig: I_s/I_0}
\end{figure}

Aside from the relation of the super-flash with optical thickness and the probe detuning, they also acquired a nice expression for the phase shift $\theta_{s}$ of the forward-scattered field $E_{s}$ relative to the incident field $E_{0}$. 

\begin{equation}
    \theta_{s} = \arccos({\frac{I_{t} - I_{0} - I_{s}}{2 \sqrt{I_{0} I_{s}}}})
\end{equation}

This phase shift angle can be summarized by their plot of $\frac{E_{s}}{E_{0}}$ in the complex plane as shown in Figure \ref{fig: theta_s}. Basically, the insight from Figure \ref{fig: theta_s} is that for a given optical thickness and temperature, as the detuning of probe laser changes, there will be various points that will represent point of normal coherent flashes or point of coherent super-flash. From the figure, they have already divided into region of super-flash (white-colored), normal flash (light grey region) and forbidden region because of energy conservation (dark grey region). They found out that at very large optical thickness, for most detunings, it will be in the super-flash region having relative phase shift $\frac{\pi}{2} \leq \theta_{s} \leq \frac{3\pi}{2}$. Meaning that there will be a number of super-flash events that could be observed when varying the detuning for a given constant high optical thickness. On the contrary, when the optical thickness is really small, the value of phase shift $\theta_{s}$ doesn't vary that much and are limited by the dashed line in Figure \ref{fig: theta_s} representing the normal coherent flash.\\

Finally,  the interesting result from \cite{Kwong2014} is the production of positive or negative super-flashes. They did this by abruptly changing the phase of the incident field $E_{0}$ using an Electro-Optical Modulator (EOM). Their result are shown in Figure \ref{fig: eom}. So, depending on how the incident field $E_{0}$ interfere with forward-scattered field $E_{s}$ after its phase has been abruptly changed, it will result in either a superflash (constructive interference) or an antiflash (destructive interference). In their case, it is noticed that when the detuning of probe field from medium resonance are negative ($\delta = -19.3 \Gamma$) as in Figure \ref{fig: eom}a, it will result in a negative superflash while when the detuning are positive ($\delta = 20.7 \Gamma$) for Figure \ref{fig: eom}b, it results in a positive superflash.

\begin{figure}[h!]
    \centering
    \includegraphics[scale = 0.7]{theta_s}
    \caption{Plot of $\frac{E_{s}}{E_{0}}$ on the complex plane. The color scale from purplish to blueish represents the probe detuning. The dark grey region, light grey region and white region represent the forbidden region (because energy conservation), the normal coherent flash region and the superflash region respectively. The solid circle and stars are the experimental data points. The transparent ellipses around the experimental data points represent the error estimate. The solid and dashed curves are the theoretical predictions for optical thickness $b_{0} = 19$ and $b_{0} = 3$ respectively. Both are at temperature $T = 3.3\mu K$. Figure courtesy of \cite{Kwong2014}.}
    \label{fig: theta_s}
\end{figure}

\begin{figure}[h!]
    \centering
    \includegraphics[scale = 1]{eom}
    \caption{Plot of normalized transmitted intensity $\frac{I_{t}(t)}{I_{0}}$ against time $t$ with an abrupt phase change of $-0.4\pi$ on $E_{0}$ at $t = 0$. a) Probe detuning $\delta = -19.3\Gamma$. b) Probe detuning $\delta = 20.7\Gamma$. The insets are the representations of $E_{0}$, $E_{s}$ and $E_{t}$ on the complex plane at the time pointed by the arrow. Figure courtesy of \cite{Kwong2014}.}
    \label{fig: eom}
\end{figure}

In summary, the 2 most important parameter in obtaining superflash effect for a pulse propagation through a cold atomic gas is;

\begin{itemize}
    \item The optical thickness $b$ must be high enough.
    \item The probe detuning $\delta$ must have certain value such that the relative phase $\theta_{s}$ between $E_{0}$ and $E_{s}$ are in the region $\frac{\pi}{2} \leq \theta_{s} \leq \frac{3\pi}{2}$ that is the superflash region based on Figure \ref{fig: theta_s}.
\end{itemize}

From these insights discussed in \cite{Kwong2014}, further research through my final year project is to acquire the maximum forward-scattered intensity $I_{s} = 4 I_{0}$ by devising a numerical simulation of the transmitted intensity for an optically thick medium. The incident probe laser will be highly-detuned from medium resonance satisfying the conditions listed in the previous summary of important points from \cite{Kwong2014}. In extension to the idea of phase-shifting the incident field $E_{0}$ (by $-0.4\pi$) as done in \cite{Kwong2014} and the next paper \cite{Kwong2015}, they devised an experiment using different kind of incident field. Instead of a one-time small phase shift to the incident field, they applied a series of periodically-spaced $\pi$ phase shift to the incident field at constant incident intensity. So, their input signal is now phase-modulated instead of amplitude-modulated. From this modification, they managed to improve the transmission quality of the phase-modulated pulse train that they had initially created. The details of their result are summarised in the next paragraphs about the original paper \cite{Kwong2015} itself.\\

In this next paper \cite{Kwong2015}, they studied the quality of the superflash train generated from phase-modulation of the incident field. The "quality" of the superflash train are quantified by the pulse contrast  $\frac{I_{c}}{I_{0}}$ and the transfer efficiency $\frac{\expval{I_{}t}}{I_{0}}$. $I_{c}$ is defined as $I_{c} = max\{I_{t}\} - \expval{I_{t}}$ where $max\{I_{t}\}$ and $\expval{I_{t}}$ are the maximum value of transmitted intensity and the mean value of transmitted intensity respectively. Also, transmitted intensity is defined as $\expval{I_{t}} = \frac{1}{T_{R}} \int_{T_{R}} \dd{t} I_{t}(t)$. Their results/plots are summarised in Figure \ref{fig: contrast}.

\begin{figure}[h!]
    \centering
    \includegraphics[scale = .8]{contrast}
    \caption{Plot of pulse contrast $\frac{I_{c}}{I_{0}}$ and transfer efficiency $\frac{\expval{I_{t}}}{I_{0}}$ against pulse repetition time $T_{R}$. a) Incident field without detuning. b) Incident field with detuning $\abs{\delta} = 11.3 \Gamma$. Red solid circle and red solid curve are experimental values and theoretical prediction of the pulse contrast respectively. The blue open circles and blue dashed curve are the experimental values and theoretical predictions of the transfer efficiency respectively. Figure courtesy of \cite{Kwong2015}.}
    \label{fig: contrast}
\end{figure}

The variable parameter is the detuning of the incident phase-modulated probe laser. The 1st case is incident field without detuning while the 2nd case is with detuning ($\abs{\delta} = 11.3 \Gamma$). The phase-modulation of the incident field are done using an Electro-Optical Modulator (EOM) in their setup as shown in Figure 1a of \cite{Kwong2015}. Their result for the transmitted signal is a series of superflashes periodically separated with each superflash peak having peak intensity of near 4 times the incident intensity as shown in Figure 1b of \cite{Kwong2015}. This result of having peak transmitted intensity closer to the classical limit of $I_{t} = 4 I_{0}$ is quite an advancement from the previous paper on superflashes where they only succeeded up to $3.1 I_{0}$ \cite{Kwong2014}.\\

When we observe for the case of incident field without detuning (Figure \ref{fig: contrast}a), the experimental values agree quite well with the theoretical prediction \cite{Kwong2015} (numerically simulated involving FFT/IFFT). There are 2 main regimes stated in \cite{Kwong2015}. The first one is the region where the repetition time are larger than the atomic excited lifetime $T_{R} >> \Gamma^{-1}$. In this region, the pulse contrast reach it steady-state constant value of $\frac{I_{c}}{I_{0}} \approx 4$. On the other hand, the transfer efficiency $\frac{\expval{I_{t}}}{I_{0}}$ approached 0. This means that most of the incident power are scattered out randomly by single atom fluorescence effect. For the second regime, when the repetition time is between coherent flash initial decay time and excited atom lifetime ($\tau_{\Bar{\nu}} \leqslant T_{R} \leqslant \Gamma^{-1}$), the pulse contrast oscillates a little bit and reached a peak value of $\approx 6$ while the transfer efficiency rapidly increase to unity. This means that the incident power are completely transferred to the transmitted pulse train even though the contrast approached 0. They concluded that in this time range, the emission from the atoms are governed by cooperativity instead of fluorescences.\\

Moving on, for the observation when the incident field are detuned from the medium's resonance by $\abs{\delta} = 11.3 \Gamma$, the experimental values also agree closely with the theoretical values as seen in Figure \ref{fig: contrast}b. For this one with detuning, in the regime that repetition time is larger than excited atomic lifetime ($T_{R} >> \Gamma^{-1}$), the behaviour of pulse contrast and transfer efficiency can be said to be more improved compared to the one without detuning. As we see in the figure, the pulse contrast achieved a higher constant value of $\approx 7.1$ compared to the one without detuning. On the other hand, the transfer efficiency also reached a higher value than the previous one with a value of $\approx 0.7$. They stated that this is due to the small optical thickness of the medium ($b_{\Bar{\nu}}(\delta) = 0.4$). Consequently, most of the transmitted intensity are in the continuous transmission mode instead of in the pulse train. Finally, for the regime where repetition time is between coherent flash initial decay time and excited atom lifetime ($\tau_{\Bar{\nu}} \leqslant T_{R} \leqslant \Gamma^{-1}$), the values of pulse contrast and transfer efficiency are not any better than the one without detuning. For example, the peak value of pulse contrast is smaller ($\approx 5$) than the one without detuning ($\approx 6$).\\

To summarize this literature review, we see that researchers had explored many ways of improving the peak value of the coherent superflash intensity. Initially, it was only up to 1 times the incident intensity \cite{Chalony2011}. Then, it was increased to about $3.1$ times the incident intensity by using a detuned incident probe laser \cite{Kwong2014}. In addition to that, in \cite{Kwong2014}, they found that the superflash value are limited to $4$ times the incident intensity and this limit can be achieved by implementing a medium with higher $OD$ at the corresponding maximal detuning $\Delta_{max}$. Subsequently, the approach to increase the peak value of superflash continued with a more unique method using a detuned, series of pulse train that is phase-modulated \cite{Kwong2015} instead of single amplitude-modulated pulse as been done previously \cite{Chalony2011, Kwong2014}. So, for this project, because we only focus on the single, amplitude-modulated square pulse and there are already knowledge on how to maximize the superflash, we instead research for a way to "control" the said superflash. In this project, we studied the superflash phenomena in a 3-level Electromagnetically Induced Transparency (EIT) medium. This EIT effect is achieved by introducing another incident laser known as the incident coupling laser that have a very crucial parameter known as the Rabi frequency $\Omega_{c}$. This said parameter will play an important role on the tunability/controllability of the superflash.
