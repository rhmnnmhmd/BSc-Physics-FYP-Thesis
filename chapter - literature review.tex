\chapter{Literature Review}
As an overview of what will be mostly covered in this literature review, some of the recently studied phenomena related to coherent light-matter interaction will be explained briefly in this section. Because this review will focus more on transient behaviour of light and how this transient component of transmitted light interacts with the main signal, we will first need to understand one of the earliest ideas/studies that have been done by researchers regarding this concept. For example, one of the earliest study regarding the unique behaviour of transient light is in the paper by Segard, Zemmouri and Macke \cite{Segard_1987} where they succeeded in generating transmitted electromagnetic pulse that that have amplitude that is slightly larger than its input amplitude. At that time this result was very counter-intuitive because we would usually expect that a light that is transmitted through a medium will have its output amplitude bounded by its input amplitude whereas this wasn't the case in their experiment.\\

In addition to that, in this literature review, we will also learn more on this idea of "amplified" transmitted light by reviewing other recent papers that utilize this concept in their experiment and papers that have succeeded in associating this behaviour of amplified transient light with other existing concept. As an example, in \cite{jeong2010slow}, they managed to generate coherent stacked optical transients and identify it as optical precursors which was a concept firstly introduced by Sommerfeld and Brillouin when they were studying the behaviour of step-modulated light through a dispersive medium to show that information in a signal cannot travel faster than speed of light in vacuum even in anomalous dispersive regime \cite{brillouin1969wave}. These kinds of paper are really interesting because we could see how certain unique physical behaviour of light that was observed in an experimental setup actually are related to a theoretical concept that was devised years before then.\\

Last but not least, we will also look at some researches that had published about other kinds of observation of optical precursors such as in \cite{Du2008} where they had observed optical precursors at the scale of bi-photon. They could identify that what they had observed is indeed an optical precursor by utilizing a widely known approximation method (stationary-phase approximation) where they could conclude that the spike that they observed in their plot is a result of interference between high- and low- frequency spectrum in the optical precursor itself. Then, we will also look at some recent development that have been done on optical precursor theory for example in \cite{Chen2010} where they have discovered that 2 optical transient phenomena that have been regarded as different phenomena, free-induction decay and optical precursor are actually the same physical process  but different realizations of the process in different conditions.\\

In summary, the goal of this literature review is to review and compile some of the researches that had been done on coherent transient light-matter interaction e.g. free-induction decay (FID)(has been successfully identified as optical precursor in \cite{Chen2010}), and especially superflash phenomena \cite{Kwong2014}. We'll review it systematically starting from its inception up to the most recent studies that have been done on it. We'll also discuss some if not all of the details in these researches from their experimental methods, what method did they use in analyzing their data; their results; and future advancement and development that have been proposed by each researcher for example on how these knowledge about coherent transients can be successfully applied to real world situation: medicine, manufacturing industry, etc.\\

Continuing from the topic of coherent transient phenomena, now, we will finally focus on the main idea for this literature review that is about coherent superflash phenomena. This phenomena is quite similar to the optical precursor phenomena in the fact that both is an ultra-fast phenomena lasting only for a short amount of time. In addition to that, the methods used in recent/most papers (on coherent flash or superflash) in comparing between experimental and numerical results are also using FFT/IFFT similar to the ones used in the various papers that have been cited in the previous section regarding numerical FFT. So, coherent transient phenomena (super-flash) will be the main focus of this project transitioning from the idea of precursors/forerunners that was introduced in some previous sections before this. We'll review some of the recent papers that have been published in the previous 5 to 7 years regarding the phenomena of coherent flash of light, more specifically the super-flash effect.\\

The study of coherent flash is quite similar to the study of optical precursor because both are coherent transient phenomena that occurs and last for only an extremely short amount time. One of the earlier paper that studied this coherent flash phenomena is in \cite{Chalony2011}. In this paper, they sent a laser light that is on-resonance with the target medium and discovered a coherent flash (almost 100\% of incident signal) that occurred when the incident laser light is suddenly switched on/off, basically having a step-on or step-off modulation similar to typical modulation used in optical precursor studies. This was really surprising because typically when a coherent light beam (laser) are shined on a medium, most of it will be scattered around randomly by the in-homogeneity in the medium and only some of it will be transmitted through the material with transmitted intensity obeying Beer-Lambert's law;

\begin{align}
    I_{t}(z) &= I_{0} e^{- \alpha_{0} z}\\
    OD &= \alpha_{0} z
\end{align}

Where $I_{t}(z)$, $I_{0}$, $OD$, $L$ and $\alpha_{0}$ are respectively the transmitted intensity, initial intensity, optical depth, medium thickness and resonant absorption coefficient of the medium \cite{Chalony2011}. For their experiment, they managed to observe the coherent flash signal by utilizing the FID mechanism that is the emission of coherent light after the incident signal is abruptly switched off. Basically, what they had done is to decrease the amount of destructive interference between incident field $E_0$ and forward scattered field $E_s$ to achieve increased intensity in the transmitted field $E_t$. One important thing is that the medium used must be optically thick to produce significant coherent flash because this coherent flash comes from the collective radiation of the dipoles in the medium that doesn't instantaneously decay when the incident field are switched off.\\

Even then, one of the main problem encountered by researchers in the observation of coherent flash is the extremely short time scale at which it exists (in the range of fsec or psec). Fortunately, it is found out that this problem can be overcome by using medium with resonant scatterers with high excited lifetime. Consequently, the phenomena of coherent flash was first observed and reported in a Nuclear Magnetic Resonance (NMR) experiment in \cite{Hahn1950} while in the optical domain, it was first observed in \cite{Brewer1972, Foster1974}. In \cite{Chalony2011}, they focused their study on FID that is causing the coherent flash in an optically thick medium and finding out how the coherent flash (its characteristic decay time) vary with optical thickness and temperature. The 3 main highlights from the results in \cite{Chalony2011} is;

\begin{itemize}
    \item Relationship of coherent flash decay time $\tau$ with rms atomic velocity $\Bar{\nu}$ (related to temperature)
    \item Relationship of coherent flash decay time $\tau$ with optical thickness $b$
    \item How relative phase $\phi$ between input field $E_{0}(t, z)$ and transmitted field $E_{t}(t, z)$ changes over time for different optical depth
\end{itemize}

For the first and second highlight of their papers can be "summarized" from their plot of decay time $\tau$ against rms atomic velocity $\Bar{\nu}$ and against optical thickness $b$ as can be seen in Figure \ref{fig: decay time}. 

\begin{figure}[h!]
    \centering
    \includegraphics[scale = 0.8]{decaytime}
    \caption{In the figure, red solid symbols correspond to the data points for abruptly switch-off experiment while the blue open symbols is for the abruptly switch-on experiment. The dashed line curve is the theoretical prediction based on Equation 4 in \cite{Chalony2011} while the solid black curve is the curve fit based on experimental data. a) Plot of coherent flash decay time $\tau$ against (inverse) rms atomic velocity $\Bar{\nu}$ at constant optical thickness of 1.2 - 1.6. b) Plot of coherent flash decay time $\tau$ against optical thickness $b$ at temperature  $T = 1.0(2) \mu K$ (circle symbols) and $T = 3.8(4) \mu K$ (triangle symbols). Figure courtesy of \cite{Chalony2011}.}
    \label{fig: decay time}
\end{figure}

We can see in Figure \ref{fig: decay time}a that the plot of coherent flash decay time from experimental data (solid black line) agree quite well with the theoretical prediction (dashed line) based on Equation 4 in \cite{Chalony2011} even though the theoretical prediction values are a little bit higher than the experimental values. This occurs because in reality, the coherent flash actually decays faster (higher decay rate thus shorter lifetime) compared to the theoretical prediction. So, because of this, every decay time acquired/observed experimentally will always be shorter than the theoretical value. In conclusion, because the rms atomic velocity $\Bar{\nu}$ is proportional to temperature of the medium $T$, we can say that the decay time $\tau$ is inversely proportional to the medium's temperature $T$.This is also the same case for Figure \ref{fig: decay time}b. We can say that the decay time of the coherent flash is inversely proportional to the optical thickness as shown by Equation 5.48, 5.49, 5.50 in \cite{Kwong2017}. As discussed in there, the initial decay time of the coherent flash are composed of 2 main factors. The first factor is a common factor of $\frac{2}{\Gamma b_{0}(0)}$ in all 3 equations (Equation 5.48 - 5.50 in \cite{Kwong2017}). This first factor is characteristic of the medium itself and doesn't depend on the type of light pulse incident on it. The 2nd factor is a factor that is different for each 3 equations because it depends on the type of modulation applied on the incident field. Therefore, only from the first factor, it is indeed true that the coherent flash decay time for any type of incident pulse is indeed inversely proportional to the optical thickness.\\

Next, they also managed to study the time-evolution of relative phase angle (between input field $E_{in}$ and output transmitted field $E_{out}^{on}(t)$) for different optical depth $b$. They did this by calculating the relative phase shift $\phi(t)$ by making use of Equation 4 in \cite{Chalony2011}. From Figure \ref{fig: relative phase}, they concluded that for each optical thickness, the relative phase shift is very small at time scale smaller than the coherent flash decay time. But, at time scale longer than decay time, the value of the relative phase shift start to become significant especially with high optical depth medium. For each optical depth, the relative phase shift increases over time until it reached its constant value. Regarding the relative phase shift discussed here, because it is not predicted by the theory that they had developed, they pointed out some possible reason for its occurrence in their experiment and potential for it to be further researched in the future. Their 1st suggestion is the relative phase shift are caused by experimental defects for example the probe laser might be slightly off-resonance with respect to the medium resonant frequency. Their other suggestion might be more interesting involving the possibility of invalidity of the application of ISA in their experiment because their medium is not dilute enough.\\

So, to summarize, some key points from their research that could be of use to my project (involving amplitude modulation of incident field) is the results regarding the relative phase shift $\phi$ between the input field $E_{in}$ and the output field $E_{out}$. Their result can be further extended from a step-on/off input field that is amplitude-modulated to a step-on/off field with phase modulation. This idea actually has been done in the next few years after their paper \cite{Chalony2011} has been published (\cite{Kwong2014} and \cite{Kwong2015}). In \cite{Kwong2014}, they applied a one-time phase shift of $-0.4\pi$ at some specific time on a detuned incident field that is propagating through an optically thick cold strontium atomic gas. As a result, they managed to get a transmitted field with intensity of about 2.5 times the incident intensity which was then called as the superflash phenomena. Another unique characteristic of the coherent flash that they had discovered from this one-time phase-shifting  of the incident field $E_{0}$ is the appearance of superflash (peak transmitted intensity higher than 100\% of incident intensity) and anti-flash (peak transmitted intensity lower than 100\% of incident intensity). This are discussed in the next few paragraphs that will be discussing the said \cite{Kwong2014} paper. Next, from \cite{Kwong2015}, instead of an amplitude modulation scheme, they implemented the idea of phase-modulation on the incident field $E_{0}$. They explored 2 idea, the first one is more simple where they only applied a single $\pi$ phase shift to the incident field $E_{0}$ while for the second one, they applied a series of periodically-spaced $\pi$ phase shifts to the incident field. This idea produced an exceptionally interesting result where they succeeded in generating a series of superflashes each with peak intensity close to the upper bound of $4 I_{0}$. For more details, this are also discussed in the next few paragraphs when discussing the specific paper \cite{Kwong2015}.\\

Later, in the next few years after the paper \cite{Chalony2011} have been published, a new study on the super-flash phenomena that is a special case of coherent flash phenomena was published in \cite{Kwong2014}. This later paper focused more on the amplitude of the transmitted field instead of its decay time. They also shined laser light on an optically thick cold strontium atomic gas like in \cite{Chalony2011}. They succeeded in observing a transmitted field with peak intensity more than 3 times the incident intensity. They concluded that this observed phenomena is possible because of the cooperative forward emission of the atoms in the target medium. The main difference between experiment in \cite{Kwong2014} and \cite{Chalony2011} that allows the observation of this super-flash phenomena is that the former varied the detuning of the probe laser with respect to the medium resonance while the latter didn't. Because of this variable detuning of incident field $E_{0}$ with respect to the medium resonance, they discovered that at certain detuning ($\abs{\delta} = 11.2 \Gamma$ in their experiment), they managed to get a transmitted signal with peak intensity about 3 times higher than incident intensity as in Figure \ref{fig: superflash}. The peak intensity for this super-flash is related to the FID mechanism involved in coherent flash. Basically, in coherent flash phenomena, the transmitted field $E_{t}$ are expressed as;

\begin{equation}
    E_{t} = E_{0} + E_{s}
\end{equation}

\begin{figure}[h!]
    \centering
    \includegraphics[scale = 0.9]{phaseshift}
    \caption{Plot of relative phase shift between input field $E_{in}$ and output transmitted field $E_{out}^{on}(t)$ against time for optical thickness of 0.3, 1.2 and 3.5 corresponding to blue circle, red cross and black square. Horizontal lines are the line for constant value of the relative phase at $t \rightarrow \infty$ for each optical thickness. Figure courtesy of \cite{Chalony2011}.}
    \label{fig: relative phase}
\end{figure}

\begin{figure}[h!]
    \centering
    \includegraphics[scale = 1]{superflash}
    \caption{Plot of normalized transmitted intensity $\frac{I_{t}(t)}{I_{0}}$ against time for detuning $\delta = 11.2\Gamma$. Red curve represents incident square pulse intensity. The black curve represents experimental data. The blue line is the level for $\frac{I_t}{I_0}$. The green open circle represents the value of $\frac{I_{s}}{I_{0}}$. Figure courtesy of \cite{Kwong2014}.}
    \label{fig: superflash}
\end{figure}

Where $E_{t}$, $E_{0}$ and $E_{s}$ is the transmitted field, incident field and forward-scattered field respectively. For an on-resonance incident field, immediately before abrupt probe laser switch-off ($t = 0^{-}$), $E_{t} \approx 0$. So, $E_{s} = - E_{0}$. Then, immediately after probe switch-off, $E_{0} = 0$. So now, $E_{t} = E_{s}$. Therefore, combining this 2 resulting expression, we will get that $E_{t} = - E_{0}$ therefore giving us $I_{t} = I_{s}$. So, in the transient region immediately before and after probe switch-off, the transmitted intensity is directly related to the forward scattered intensity and in this case, equal to the forward scattered intensity. Therefore, this direct relationship between $I_{t}$ and $I_{s}$ in the region before/after abrupt probe switch-off is what allows for the transmitted intensity in Figure \ref{fig: superflash} to exceed the incident intensity when the incident field is detuned by a certain amount as shown in Figure \ref{fig: detuned}. 

\begin{figure}[h!]
    \centering
    \includegraphics[scale = 1]{detuned}
    \caption{Plot of normalized forward-scattered intensity against probe laser detuning. The green solid dots are experimental data. The black solid line is the theoretical prediction. Figure courtesy of \cite{Kwong2014}.}
    \label{fig: detuned}
\end{figure}

In addition to this, by conservation of energy, they showed that the forward-scattered intensity $I_{s}$ is limited by 4 times the incident intensity $I_{0}$. It is simply from the fact that;

\begin{equation}
    E_{t} = E_{0} + E_{s} \leq E_{0}
\end{equation}

Taking the modulus-squared of both side of the equation;

\begin{equation}
    \abs{E_{0} + E_{s}}^{2} \leq \abs{E_{0}}^{2}
\end{equation}

They interpreted this as the allowed value of $E_{s}$. The allowed value of $E_{s}$ will be in a circle of radius $\abs{E_{0}}$ centered at $-E_{0}$. Therefore, the maximum value of $E_{s}$ is $E_{s} = -2 E_{0}$. So, taking the square of both sides for intensity they finally get upper bound for the forward-scattered intensity, $I_{s} \leq 4 I_{0}$. To show this, they plotted a colored plot representing the relation of the normalized forward-scattered intensity $\frac{I_{s}}{I_{0}}$ with 2 parameters, the detuning $\delta$ and the optical thickness (at resonance) $b_{0}$ as shown in Figure \ref{fig: I_s/I_0}. As we can clearly see from Figure \ref{fig: I_s/I_0}, when the white line that represents super-flash occurrence are extrapolated, it will eventually reach the theoretical upper bound $\frac{I_{s}}{I_{0}} = 4$.\\

\newpage

\begin{figure}[h!]
    \centering
    \includegraphics[scale = 1]{IsI0}
    \caption{Colored plot of $\frac{I_{s}}{I_{0}}$  against resonant optical thickness (non-zero temperature) $b_{\bar{v}}(\delta) = b_{\Bar{v}}(0)$ and detuning $\frac{\abs{\delta}}{\Gamma}$ for the temperature $T = 3.3(2) \mu K$. Black dashed line represents optical thickness of their experiment. The white solid line represents linear dependence of the detuning $\delta$ to the resonant optical thickness $b_{\bar{v}}(0)$ for when maximum value of $\frac{I_{s}}{I_{0}}$ occurs (super-flash). Figure courtesy of \cite{Kwong2014}.}
    \label{fig: I_s/I_0}
\end{figure}

Aside from the relation of the super-flash with optical thickness and the probe detuning, they also acquired a nice expression for the phase shift $\theta_{s}$ of the forward-scattered field $E_{s}$ relative to the incident field $E_{0}$. 

\begin{equation}
    \theta_{s} = \arccos({\frac{I_{t} - I_{0} - I_{s}}{2 \sqrt{I_{0} I_{s}}}})
\end{equation}

This phase shift angle can be summarized by their plot of $\frac{E_{s}}{E_{0}}$ in the complex plane as shown in Figure \ref{fig: theta_s}. Basically, the insight from Figure \ref{fig: theta_s} is that for a given optical thickness and temperature, as the detuning of probe laser changes, there will be various points that will represent point of normal coherent flashes or point of coherent super-flash. From the figure, they have already divided the region into region of super-flash (white-colored), normal flash (light grey region) and forbidden region because of energy conservation (dark grey region). They found out that at very large optical thickness, for most detunings, it will be in the super-flash region having relative phase shift $\frac{\pi}{2} \leq \theta_{s} \leq \frac{3\pi}{2}$. Meaning that there will be a number of super-flash events that could be observed when varying the detuning for a given constant high optical thickness. On the contrary, when the optical thickness is really small, the value of phase shift $\theta_{s}$ doesn't vary that much and are limited by the dashed line in Figure \ref{fig: theta_s} representing the normal coherent flash.

\begin{figure}[h!]
    \centering
    \includegraphics[scale = 0.7]{theta_s}
    \caption{Plot of $\frac{E_{s}}{E_{0}}$ on the complex plane. The color scale from purplish to blueish represents the probe detuning. The dark grey region, light grey region and white region represent the forbidden region (because energy conservation), the normal coherent flash region and the superflash region respectively. The solid circle and stars are the experimental data points. The transparent ellipses around the experimental data points represent the error estimate. The solid and dashed curves are the theoretical predictions for optical thickness $b_{0} = 19$ and $b_{0} = 3$ respectively. Both are at temperature $T = 3.3\mu K$. Figure courtesy of \cite{Kwong2014}.}
    \label{fig: theta_s}
\end{figure}

Finally,  the final interesting result from \cite{Kwong2014} is the production of positive or negative super-flashes. They did this by abruptly changing the phase of the incident field $E_{0}$ using an Electro-Optical Modulator (EOM). Their result are shown in Figure \ref{fig: eom}. So, depending on how the incident field $E_{0}$ interfere with forward-scattered field $E_{s}$ after its phase has been abruptly changed, it will result in either a superflash (constructive interference) or an antiflash (destructive interference). In their case, it is noticed that when the detuning of probe field from medium resonance are negative ($\delta = -19.3 \Gamma$) as in Figure \ref{fig: eom}a, it will result in a negative superflash while when the detuning are positive ($\delta = 20.7 \Gamma$) for Figure \ref{fig: eom}b, it results in a positive superflash.

\newpage

\begin{figure}[h!]
    \centering
    \includegraphics[scale = 1]{eom}
    \caption{Plot of normalized transmitted intensity $\frac{I_{t}(t)}{I_{0}}$ against time $t$ with an abrupt phase change of $-0.4\pi$ on $E_{0}$ at $t = 0$. a) Probe detuning $\delta = -19.3\Gamma$. b) Probe detuning $\delta = 20.7\Gamma$. The insets are the representations of $E_{0}$, $E_{s}$ and $E_{t}$ on the complex plane at the time pointed by the arrow. Figure courtesy of \cite{Kwong2014}.}
    \label{fig: eom}
\end{figure}

In summary, the 2 most important parameter in obtaining superflash effect for a pulse propagation through a cold atomic gas is;

\begin{itemize}
    \item The optical thickness $b$ must be high enough.
    \item The probe detuning $\delta$ must have certain value such that the relative phase $\theta_{s}$ between $E_{0}$ and $E_{s}$ are in the region $\frac{\pi}{2} \leq \theta_{s} \leq \frac{3\pi}{2}$ that is the superflash region based on Figure \ref{fig: theta_s}.
\end{itemize}

From these insights from \cite{Kwong2014}, further research through my final year project could be done for example on the acquirement of the maximum forward-scattered intensity $I_{s} = 4 I_{0}$ by devising a numerical simulation of the transmitted intensity for a more optically thick medium probed by a laser that is highly detuned from the medium resonance satisfying the conditions needed based on Figure \ref{fig: I_s/I_0}. In extension to the idea of phase-shifting the incident field $E_{0}$ (by $-0.4\pi$) as done in \cite{Kwong2014}, in the next paper \cite{Kwong2015}, they devised an experiment using different kind of incident field. Instead of a one-time small phase shift to the incident field, they applied a series of periodically-spaced $\pi$ phase shift to the incident field at constant incident intensity. So, their input signal is now phase-modulated instead of amplitude-modulated. From this modification, they managed to improve the initial decay time of the superflash (making the decay time to be longer) and also improve the transmission quality of the phase-modulated pulse train that they had initially created. The details of their result are discussed in the next few paragraphs about the original paper \cite{Kwong2015} itself.\\

In this next paper \cite{Kwong2015}, firstly, they studied the dependence of initial decay time of coherent flash $\tau_{\Bar{\nu}}(\delta)$ on probe detuning $\delta$ for 2 different kind of incident field. The first type of incident field is like a typical FID experiment where the incident field are abruptly switched off by modulating its amplitude. The second type of incident field is where they abruptly change the phase (by $\pi$) of the incident field  while letting the amplitude to stay constant. This abrupt phase change of the incident field are done using an Electro-Optical Modulator (EOM) in their setup as shown in Figure 1a of \cite{Kwong2015}. Their plot of decay time against detuning can be seen in Figure \ref{fig: decaytimedetuning}.

\begin{figure}[h!]
    \centering
    \includegraphics[scale = 0.8]{decaytimedetuning}
    \caption{Plot of initial decay time of coherent flash against the probe detuning at $k \Bar{\nu} = 3.4\Gamma$. The blue circles and curves are for the abrupt amplitude switch-off (FID) experiment while the red squares and curves are for the abrupt ($\pi$) phase change experiment. The blue open circle and blue solid curve are experimental data points and theoretical prediction respectively. The red squares and red dashed curve are experimental data points and theoretical prediction respectively. The blue dotted curve and the red dash-dotted curve are numerical predictions respectively (including the finite response time of the experimental setup. Figure courtesy of \cite{Kwong2015}.}
    \label{fig: decaytimedetuning}
\end{figure}

\newpage

We can see from Figure \ref{fig: decaytimedetuning} for the abrupt switch off experiment (FID) that the experimental data points (blue circles) are in acceptable agreement with the theoretical predictions (blue solid curve) even though it have large statistical error because of finite response time of their experimental setup. The theoretical prediction was obtained using Equation 5 of \cite{Kwong2015}. To correct for the finite response time, they numerically simulated the transmitted signal involving FFT/IFFT and extracted the initial decay time. From this numerical procedure, they got a better agreement between experimental data (blue open circles in Figure \ref{fig: decaytimedetuning}) and numerical data (blue dotted curve in Figure \ref{fig: decaytimedetuning}). All of these are then also repeated for the abrupt ($\pi$) phase change experiment with constant incident intensity. As we can see from Figure \ref{fig: decaytimedetuning}, this time, the 2 experimental data points that they had acquired (red squares) have a significantly large deviation from the theoretical prediction (red dashed curve) that was plotted based on Equation 6 in \cite{Kwong2015}. This is also because of finite response time of the EOM that they had used to apply the abrupt phase change. Then, applying the same procedure as done for the FID experiment, they numerically simulated the transmitted signal for incident field with abrupt phase change and extracted the initial decay times to plot the numerical prediction (red dash-dotted curve in Figure \ref{fig: decaytimedetuning}). Now, the experimental results and numerical results agree really well compared to the previous one.\\

The next main result from \cite{Kwong2015} is their experiment applying the abrupt $\pi$ phase jump that was mentioned previously as a periodic modulation on an input incident field. Their result for the transmitted signal is a series of superflashes periodically separated with each superflash peak having peak intensity of near 4 times the incident intensity as shown in Figure 1b of \cite{Kwong2015}. This result having transmitted intensity closer to the classical limit of $I_{t} = 4 I_{0}$ is quite an advancement from the previous paper on superflashes where they only succeeded up to $3.1 I_{0}$ \cite{Kwong2014}.\\

Continuing from the same type of phase-modulated series of periodic square pulses, next, they investigated the relation of two pulse characteristics that they had introduced with the repetition time of the incident field $T_{R}$ for incident field without detuning and with detuning ($\abs{\delta} = 11.3 \Gamma$). These 2 new transmitted pulse characteristic are the pulse contrast $\frac{I_{c}}{I_{0}}$ and transfer efficiency $\frac{\expval{I_{}t}}{I_{0}}$. $I_{c}$ is defined as $I_{c} = max\{I_{t}\} - \expval{I_{t}}$ where $max\{I_{t}\}$ and $\expval{I_{t}}$ are the maximum value of transmitted intensity and the mean value of transmitted intensity respectively. Also, transmitted intensity is defined as $\expval{I_{t}} = \frac{1}{T_{R}} \int_{T_{R}} \dd{t} I_{t}(t)$. Their results/plots are summarised in Figure \ref{fig: contrast}.

\begin{figure}[h!]
    \centering
    \includegraphics[scale = .8]{contrast}
    \caption{Plot of pulse contrast $\frac{I_{c}}{I_{0}}$ and transfer efficiency $\frac{\expval{I_{t}}}{I_{0}}$ against pulse repetition time $T_{R}$. a) Incident field without detuning. b) Incident field with detuning $\abs{\delta} = 11.3 \Gamma$. Red solid circle and red solid curve are experimental values and theoretical prediction of the pulse contrast respectively. The blue open circles and blue dashed curve are the experimental values and theoretical predictions of the transfer efficiency respectively. Figure courtesy of \cite{Kwong2015}.}
    \label{fig: contrast}
\end{figure}

\newpage

Firstly, when we observe for the case of incident field without detuning (Figure \ref{fig: contrast}a), the experimental values agree quite well with the theoretical prediction that was acquired using Equation 1 and 2 in \cite{Kwong2015} (numerically simulated involving FFT/IFFT). There are 2 main regimes that we can notice as stated by Kwong et al in \cite{Kwong2015}. The first one is the region where the repetition time are larger than the atomic excited lifetime $T_{R} >> \Gamma^{-1}$. In this region, the pulse contrast reach it steady-state constant value of $\frac{I_{c}}{I_{0}} \approx 4$. On the other hand, the transfer efficiency $\frac{\expval{I_{t}}}{I_{0}}$ approached 0. This means that most of the incident power are scattered out randomly by single atom fluorescence effect. For the second regime, when the repetition time is between coherent flash initial decay time and excited atom lifetime ($\tau_{\Bar{\nu}} \leqslant T_{R} \leqslant \Gamma^{-1}$), the pulse contrast oscillates a little bit and reached a peak value of $\approx 6$ while the transfer efficiency rapidly increase to unity. This means that the incident power are completely transferred to the transmitted pulse train even though the contrast approached 0. They concluded that in this time range, the emission from the atoms are governed by cooperativity instead of fluorescences.\\

Moving on, for the observation when the incident field are detuned from the medium's resonance by $\abs{\delta} = 11.3 \Gamma$, the experimental values also agree closely with the theoretical values as seen in Figure \ref{fig: contrast}b. For this one with detuning, in the regime that repetition time is larger than excited atomic lifetime ($T_{R} >> \Gamma^{-1}$), the behaviour of pulse contrast and transfer efficiency can be said to be more improved compared to the one without detuning. As we can see in the figure, the pulse contrast achieved a higher constant value of $\approx 7.1$ compared to the one without detuning. On the other hand, the transfer efficiency also reached a higher value than the previous one with a value of $\approx 0.7$. They stated that this is due to the small optical thickness of the medium ($b_{\Bar{\nu}}(\delta) = 0.4$). Consequently, most of the transmitted intensity are in the continuous transmission mode instead of in the pulse train. Finally, for the regime where repetition time is between coherent flash initial decay time and excited atom lifetime ($\tau_{\Bar{\nu}} \leqslant T_{R} \leqslant \Gamma^{-1}$), the values of pulse contrast and transfer efficiency are not any better than the one without detuning for example the peak value of pulse contrast is smaller ($\approx 5$) than the one without detuning ($\approx 6$).\\

In summary, initial decay time of the coherent flash weakly depends on the detuning of incident field $\delta$ and depends mainly on the optical thickness of the medium $b_{\Bar{\nu}}(\delta) = b_{0}(0)$ (at resonance and zero temperature). In addition to that, the quality of transmitted train of pulse in terms of the pulse contrast and the transfer efficiency are better with a detuned incident field that have repetition time larger than the excited atomic lifetime. Next, in terms of further ideas that could be explored from this paper in this final year project might be on changing the type of modulation for the series of incident square pulses instead of using a $\pi$ phase jump modulation, an amplitude modulation scheme could be explored for potential new discovery or improvement. Even if the amplitude modulation turns out to be inferior to the phase modulation scheme, it might still be possible to discover some sort of upper/lower bound on the incident field's or medium's parameters that could optimize the quality of the transmitted field compared to the other/previous amplitude-modulation scheme.
