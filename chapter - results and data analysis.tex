\chapter{Results and Data Analysis}\label{results and analysis}
As this thesis are focused on coherent light-matter interactions specifically the coherent superflash phenomena, the results and analysis that will be presented in this chapter will be crucially related to that phenomena.\\

The results and analysis that will be presented in this chapter will be heavily focused on the extension of coherent superflash research in a new kind 3-level EIT-enabled medium. The idea of extending the research of coherent superflash into 3-level EIT-enabled medium mainly came from these 2 research papers one published in Optic Letters and the other in Physical Review Letters. The first one that was published in Optics Letters was not actually researching on the coherent superflash like this project but actually they researched about the interference phenomena between the optical precursors and the delayed main field (slow-light effect caused by EIT) \cite{Jeong2010}. The main takeaway from their research are the explicit expression for the 3-level EIT-enabled medium and the numerical FFT/IFFT technique they'd implement to study transmitted field/s temporally & spectrally. Meanwhile, the second paper \cite{Kwong2014} was actually researching the coherent superflash that's definitely related to this project. They observed the formation of superflash for light incident on an optically thick cold strontium gas both experimentally and numerically. The main idea from their paper that is incorporated in this project is on how they studied the coherent superflash by studying the behaviour of the forward-scattered intensity (the superflash itself) when the laser probe detuning are varied (Refer Figure 1d of \cite{Kwong2014}).

\newpage

So, in this chapter, it will be divided to 4 sections. The 1st section in this chapter will be dedicated to showing some temporal profiles that was generated using the programming code that was developed throughout this project. Its purpose is to show the validity of the code and how it's able to produce accurate & similar temporal profiles to the ones produced in many previous research papers. Next, for the 2nd section in this chapter, the focus will shift to the forward-scattered intensity spectrum (Forward-scattered intensity $I_{s}$ vs Probe Laser detuning $\Delta$). The difference between the forward-scattered intensity spectrum that was produced from the usual 2-level medium and the 3-level EIT-enabled medium will be emphasized here. We'll see a new behaviour around the resonances region that can be counted as a new finding from this project because as of right now, there has never been any studies on the forward-scattered intensity spectrum for 3-level EIT-enabled medium. Going further, in the 3rd section, the newly-discovered pattern of the forward-scattered spectrum for 3-level EIT medium will be justified by comparing its peak values (near-resonance and off-resonance) with their corresponding temporal profile. Finally, in the 4th section, the effect of varying the coupling laser's Rabi frequency on the forward-scattered intensity spectrum will be shown through the different output plots. We then propose an explanation for this effect by relating it to the detuning of the incident field $\Delta$ and the phase shift of the transmitted field $\phi(\delta)$.

\newpage

\section{Reproduction of Temporal Profiles}
\begin{figure}[h!]
    \centering
    \begin{subfigure}[h!]{0.5\textwidth}
         \centering
         \includegraphics[width = \textwidth]{oriPlot}
         \caption{Figure courtesy of \cite{jeong2010slow}}
         \label{fig: original plot}
    \end{subfigure}
    
    \hfill
    
    \begin{subfigure}[h!]{0.5\textwidth}
         \centering
         \includegraphics[width = \textwidth]{Temp_Profile_EIT_On-Resonance}
         \caption{Temporal profile generated from Python code developed in this project.}
         \label{fig: code plot}
    \end{subfigure}
    
    \caption{Comparison of the transmitted intensity temporal profile for square light pulse incident on 3-level EIT-enabled medium.}
    \label{fig: comparing temporal profile}
\end{figure}

From Figure \ref{fig: comparing temporal profile}, we can see that the Python code developed through this project were able to simulate the transmission intensity temporal profile of \cite{jeong2010slow} successfully. This serve to show that the simulation code developed from this project (More details discussed in the Methodology chapter) are valid and reliable enough to be utilized as the main tool for researching the coherent superflash phenomena.\\

The code for simulating the transmission temporal profile are really crucial for this project and must be sufficiently accurate because all further results/plots after this heavily depends on the temporal profile code. This is because, for example, for the next result section that will be discussing the forward-scattered intensity spectrum $I_{s}$, the plots displayed in that section (values of $I_{s}$) are extracted at the abrupt extinction point in the temporal profile varied over different values of probe detuning $\Delta$. Therefore, if Figure \ref{fig: original plot} can't even be properly reproduced with correct values by the code developed here, then all further outputs would be pointless.\\

Analysis of the plot in Figure \ref{fig: original plot} have been done in details in terms of optical precursors and delayed main field \cite{jeong2010slow} but they didn't focus specifically on the coherent superflash that occurred at the abrupt-extinction region. In terms of coherent superflash studies, the (peak) value of transmitted intensity at the abrupt-extinction region is exactly equivalent to the forward-scattered intensity $I_{s} = |E_{s}|^{2}$. The mechanism for how the value of this forward-scattered intensity $I_{s}$ varies with probe detuning $\Delta$ are discussed in detail in \cite{Chalony2011, Kwong2014, Kwong2017}. Some of their main points are also discussed in the Literature Review chapter. In short, the value of forward-scattered intensity $I_{s}$ depends on the phase of its field $\phi_{s}$ that also depends on the detuning of the probe field $\Delta$. Therefore, certain detuning of the probe field can produce forward-scattered field with a phase such that it will have a magnitude of more than 1 (basically superflash phenomena) at the abrupt-extinction point.\\

The Figure \ref{fig: theta_s} that was taken from \cite{Kwong2014} summarized this idea concisely where we can notice that the magnitude of the forward-scattered field are not necessarily bounded to the magnitude of the incident electric field. In addition to that, the interference between the incident field and forward-scattered field doesn't have anything to do with the values of the forward-scattered intensity at the abrupt-extinction point. This is true because during that point in time, the probe laser has been turned off and the forward-scattered field are actually radiated by the "residual" dipoles in the medium that doesn't instantaneously decay to $0$.\\

This is also related to the cooperative emission effect that was mentioned in \cite{Kwong2014, Kwong2015, Kwong2017, Araujo2016} that is prevalent in optically thick medium. Its main effect is to reduce the decay time (inverse of atomic excited-state decay rate $\Gamma$) of the superflash itself by a certain different factors depending on the modulation of the incident field. The specific expression for these factors can be found in \cite{Kwong2017} and they also mentioned that there are a common factor even for different type of modulation that doesn't depend on the temperature of the medium or even the detuning of the probe field. The factor is $\frac{2}{b_{0}(0)}$ where $b_{0}(0)$ are the resonant optical depth of the medium at zero-temperature which is the main parameter that governs this cooperative effect.\\

To demonstrate the dependence of the forward-scattered intensity $I_{s}$ to the probe detuning $\Delta$, some temporal profiles are shown in Figure \ref{fig: comparing temporal profile} but instead of 0-detuning case as produced by \cite{jeong2010slow} as shown in Figure \ref{fig: original plot}, the temporal profiles are for the case of non-zero detuning. Both of it are generated from the simulation code that is developed in this project.\\

As can be seen from Figure \ref{fig: comparing temporal profile}, the forward-scattered intensity $I_{s}$ (Transmission intensity immediately after abrupt extinction) does indeed vary depending on the probe detuning $\Delta$. We could get both the desired coherent superflash (Figure \ref{fig: comparing temporal profile}b and \ref{fig: comparing temporal profile}c) and the coherent antiflash also mentioned in \cite{Kwong2017} (Forward-scattered intensity lower than incident intensity. Figure \ref{fig: comparing temporal profile}a). In addition to that, the coherent superflash produced in Figure \ref{fig: comparing temporal profile}b and \ref{fig: comparing temporal profile}c both are significantly higher than the one obtained in \cite{jeong2010slow} with a value of nearly $3I_{0}$ and about $3I_{0}$ respectively. These values are approaching the upper limit value of $4I_{0}$ according to the discussion in \cite{Kwong2014}.

\begin{figure}[h!]
    \centering
    \includegraphics[scale = 0.65]{Figures/3temporalProfileDetuned.png}
    \caption{Transmission intensity temporal profile of an incident square pulse (pulse duration of $2\cdot10^{-6}$ sec) on cold, dilute atomic gas of Rb-85 with optical depth (OD) of 62 and coupling laser's Rabi frequency of $4.2\gamma_{13}$ where $\gamma_{13}$ is half of the natural linewidth $\gamma_{13} = \frac{\Gamma_{3}}{2} = 1.8850 \cdot 10^{7} \frac{rad}{sec}$. Each only have different probe detuning $\Delta$. a) $\Delta = 5.6764\gamma_{13}$. b) $\Delta = 0.3714\gamma_{13}$. c) $\Delta = 11.2467\gamma_{13}$.}
    \label{fig: comparing temporal profile}
\end{figure}

The resulting coherent superflash in both Figure \ref{fig: comparing temporal profile}b and \ref{fig: comparing temporal profile}c have peak value similar to the result obtained in Figure 1c of \cite{Kwong2014}. They obtained a peak value of about $3.1 I_{}0$. Since Figure 1c of \cite{Kwong2014} are obtained just using a 2-level medium, this doesn't really show any advantages of using 3-level EIT-enabled medium over the former. This only supports the idea that the peak value of a coherent superflash can be optimized to be closer to the upper limit value of $4I_{0}$ by detuning the probe laser by a specific amount. As shown in \cite{Kwong2014}, for a 2-level medium, the value of the coherent superflash can be made closer to $4I_{0}$ by using a medium with larger OD and detuning the probe to the optimal detuning $\Delta_{max}$. For a medium with large OD, they acquired an expression for the maximal detuning;

\begin{equation}
    |\Delta_{max}| \approx \frac{b_{0}(0) \Gamma}{4\pi} 
\end{equation}

Where;

\begin{itemize}
    \item $b_{0}(0) = \frac{b_{\Bar{v}}(0)}{g(\frac{kv}{\Gamma})}$ is the resonant OD at zero-temperature. 
    \item $b_{\Bar{v}}(0)$ is the resonant OD at non-zero temperature.
    \item $g(\frac{kv}{\Gamma})$ is a real function related to the Complementary Error Function $Erfc(x)$ \cite{abramowitz1965ia}.
\end{itemize}

In addition to that, they also showed that at this maximal detuning $\Delta_{max}$, the phase shift of the transmitted field with respect to the incident field $\phi$ will always be $\approx \pi$.\\

So, similar analysis have been done for a 3-level EIT-enabled medium to acquire an explicit expression for the maximal detuning $\Delta_{max}$. The explicit expression that was obtained here is;

\begin{equation}
    \Delta_{max} = \Delta_{max}(\Omega_{c}) ?
\end{equation}

%analysis about the above explicit expression, dependencies etc

In summary, the explicit expression for the maximal detuning $\Delta_{max}$ that was obtained here would be really useful for other researcher if they wanted to produce a transmitted pulse with the peak value of the coherent superflash.


\section{Forward-Scattered Intensity Spectrum for 3-Level EIT-Enabled Medium}
In Figure \ref{fig: comparing Is spectrum}. The $I_{s}$ spectrum for a 3-level medium was compared. The parameters and important equations (e.g. electric susceptibility $\chi$) are based on \cite{jeong2010slow}. This idea of comparing the $I_{s}$ spectrum are inspired by Figure 1d of \cite{Kwong2014} where they had plotted the $I_{s}$ spectrum for their 2-level medium while in here it is done for a 3-level EIT-possible medium. In Figure \ref{fig: comparing Is spectrum}a, even though it is a 3-level EIT-possible medium, the $I_{s}$ spectrum are similar to a 2-level medium's spectrum for example the ones plotted in Figure 1d of \cite{Kwong2014} which is a 2-level medium but for different atomic species. This is because the coupling laser's Rabi frequency $\Omega_{c}$ is set to $0$. The coupling laser's Rabi frequency $\Omega_{c}$ basically acts as a "switch" for the EIT effect. When it is "switched-off", the coupling between the 3rd new coupling energy level $\ket{c}$ with the excited-state energy level $\ket{e}$ doesn't occur therefore no EIT effect present. A nice schematic of this said energy levels configuration from \cite{Jeong2009} can be seen in Figure \ref{fig: 3LevelAtom}.

\begin{figure}[h!]
    \centering
    \includegraphics[scale = 0.65]{Figures/IsSpectrumComparison.png}
    \caption{The forward-scattered intensity detuning spectrum ($I_{s}$ spectrum from now on) for the same initial parameters but a) Without EIT (Coupling laser's Rabi frequency $\Omega_{c} = 0$) while b) With EIT (Coupling laser's Rabi frequency $\Omega_{c} = 4.2\gamma_{eg}$ where $\gamma_{eg} = 1.885 \cdot 10^{7} \frac{rad}{sec}$).}
    \label{fig: comparing Is spectrum}
\end{figure}

\begin{figure}[h!]
    \centering
    \includegraphics[scale = 1]{Figures/3LevelAtom.png}
    \caption{Note that the different notation used here for the energy levels. $\ket{1}$ is equivalent to $\ket{g}$. $\ket{2}$ is equivalent to $\ket{c}$. $\ket{3}$ is equivalent to $\ket{e}$. Figure courtesy of \cite{Jeong2009}.}
    \label{fig: 3LevelAtom}
\end{figure}

While for Figure \ref{fig: comparing Is spectrum}b, the coupling laser's Rabi frequency $\Omega_{c}$ was set to $4.2\gamma_{eg}$ where $\gamma_{eg} = \frac{\Gamma_{e}}{2} = 1.885 \cdot 10^{7} \frac{rad}{sec}$. All the other parameters for example the OD (OD = 62) are according to \cite{jeong2010slow}. As observed in Figure \ref{fig: comparing Is spectrum}b, a nearly-symmetrical fluctuations started to form near the resonance while the other parts of the spectrum remained unchanged. In addition to that, the peak value of the new fluctuations is really close to the maximum superflash value at the maximal detuning $\Delta_{max}$ (that was derived for a 2-level medium in \cite{Kwong2017}). This is a new and very interesting result from this project that had never been acquired before.\\

Through combining some crucial ideas from \cite{Kwong2014} and \cite{jeong2010slow}, we were able to obtain the new $I_{s}$ spectrum for a 3-level EIT-enabled medium. The new feature of the $I_{s}$ spectrum shown here are definitely interesting because we obtained an alternate method to produce a new superflash value (near-resonance) that is close to the peak superflash value (at maximal detuning $\Delta_{max}$). The previous method (in 2-level medium) as shown in \cite{Kwong2017} was by using a medium with larger OD because as can be seen in Fig 5.10 and 5.11 of \cite{Kwong2017}, medium with larger OD will have multiple points with superflash values close to the global maximum value. So, by using a 3-level EIT-enabled medium, a new superflash points (close to global maximum superflash value) can be created by controlling the coupling laser's Rabi frequency $\Omega_{c}$ instead of manipulating the OD of the target medium.


\section{Corresponding Temporal Profile for The Maximum Points in The 3-Level EIT Medium's Forward-Scattered Intensity Spectrum}

In this section, it is dedicated to show the validity of the $I_{s}$ spectrum that were obtained for the 3-level EIT-enabled medium using the code developed in this project by comparing the peak values in the $I_{s}$ spectrum with the peak value of transmitted intensity (which occurs at the extinction point and are equivalent to $I_{s}$) in their corresponding temporal profile. There will be 3 figures in this section. Each of it are for different values of coupling laser's Rabi frequency $\Omega_{c}$. The first, second and the third figure will have coupling laser's Rabi frequency $\Omega_{c}$ of $0$, $4.2\gamma_{eg}$ and $8.4\gamma_{eg}$ respectively.

\newpage

\begin{figure}[h!]
    \centering
    \includegraphics[scale = 0.65]{Figures/correspondingTemporalProfile2.png}
    \caption{The corresponding temporal profiles at local maxima point in the $I_{s}$ spectrum b) near resonance and c) off-resonance. a) $I_{s}$ spectrum for 3-level EIT-disabled medium ($\Omega_{c} = 0 \frac{rad}{sec}$). b)  Temporal profile at probe detuning $\Delta = 3.1565\gamma_{eg}$. c) Temporal profile at probe detuning $\Delta = 10.8488\gamma_{eg}$.}
    \label{fig: corresponding temporal profile 1}
\end{figure}

Firstly, in Figure \ref{fig: corresponding temporal profile 1}, the $I_{s}$ spectrum are for the $\Omega_{c} = 0$ case. Making it a 3-level EIT-disabled medium (which is equivalent to a 2-level medium). All other parameters e.g. OD and medium atomic species are according to \cite{jeong2010slow}. The detuning points that are chosen are the local maxima near resonance ($\Delta = 3.1565\gamma_{eg}$) and the global maxima ($\Delta = 10.8488\gamma_{eg}$). As can be clearly seen, the value of $I_{s}$ in both temporal profiles agree closely with the value of $I_{s}$ in the spectrum at their respective probe detuning. Therefore, for the 3-level EIT-disabled case here, the $I_{s}$ value in the $I_{s}$ spectrum are able to represent accurately and excellently the $I_{s}$ value in the temporal profile at the corresponding detuning.\\

Next, in Figure \ref{fig: corresponding temporal profile 2}, the $I_{s}$ spectrum are for the $\Omega_{c} = 4.2\gamma_{eg}$ case. Making it a 3-level EIT-enabled medium. All other parameters e.g. OD and medium atomic species are still kept similar to the previous case. The detuning points that are chosen are the maxima point near resonance ($\Delta = 0.3714\gamma_{eg}$. The value are really close to the global maxima's value) and the global maxima point ($\Delta = 11.2467\gamma_{eg}$). As shown in the figure, the value of $I_{s}$ in both temporal profiles also agree closely with the value of $I_{s}$ in the spectrum at their respective probe detuning. At the first point, $I_{s}$ is about $2.7I_{0}$ while at the 2nd chosen point, it's about $3I_{0}$. So, for the case here, the $I_{s}$ value in the $I_{s}$ spectrum are also able to represent the $I_{s}$ value in the temporal profile at the corresponding detuning.

\begin{figure}[h!]
    \centering
     \includegraphics[scale= 0.65]{Figures/correspondingTemporalProfile1.png}
    \caption{The corresponding temporal profiles at local maxima point in the $I_{s}$ spectrum b) near resonance and c) off-resonance. a) $I_{s}$ spectrum for 3-level EIT-enabled medium ($\Omega_{c} = 4.2\gamma_{eg}$). b) Temporal profile at probe detuning $\Delta = 0.3714\gamma_{eg}$. c) Temporal profile at probe detuning $\Delta = 11.2467\gamma_{eg}$.}
    \label{fig: corresponding temporal profile 2}
\end{figure}

Finally, in Figure \ref{fig: corresponding temporal profile 3}, the $I_{s}$ spectrum are for the $\Omega_{c} = 8.4\gamma_{eg}$ case. Making it also a 3-level EIT-enabled medium like the 2nd case. All parameters e.g. OD and medium atomic species are unchanged and identical to the previous case. The detuning points that are chosen are the maxima point near resonance ($\Delta = 1.2997\gamma_{eg}$. The value are also really close to the global maxima's value) and the global maxima point ($\Delta = 12.3077\gamma_{eg}$). As expected, the value of $I_{s}$ in both temporal profiles agree closely with the value of $I_{s}$ in the spectrum at their respective probe detuning. At both chosen points, the $I_{s}$ values are about $3I_{0}$. Hence, for the case here, the $I_{s}$ value in the $I_{s}$ spectrum are also able to represent the $I_{s}$ value in the temporal profile at the respective detuning.

\begin{figure}[h!]
    \centering
    \includegraphics[scale = 0.47]{Figures/correspondingTemporalProfile3.png}
    \caption{The corresponding temporal profiles at local maxima point in the $I_{s}$ spectrum b) near resonance and c) off-resonance. a) $I_{s}$ spectrum for 3-level EIT-enabled medium ($\Omega_{c} = 8.4\gamma_{eg}$). b) Temporal profile at probe detuning $\Delta = 1.2997\gamma_{eg}$. c) Temporal profile at probe detuning $\Delta = 12.3077\gamma_{eg}$.}
    \label{fig: corresponding temporal profile 3}
\end{figure}

In summary, for every cases presented here, each of the $I_{s}$ spectrum have the same values with the $I_{s}$ values in the corresponding temporal profiles at the selected detuning points. It can also be noted that for both the near-resonance and the off-resonance maxima points, as the coupling laser's Rabi frequency $\Omega_{c}$ is increased, their respective detuning value also increase. So, it is hypothesized that this detuning $\Delta$ are directly proportional to the coupling laser's Rabi frequency $\Omega_{c}$. This hypothesis will be discussed in more details in the next section with a proposed relevant equations.


\section{Relation between The Forward-Scattered Intensity Spectrum with The Coupling Laser's Rabi Frequency}

As mentioned in the previous section, it is observed that, when the coupling laser's Rabi frequency are switched-on and as its value are increased, aside from the production of a new fluctuations in the $I_{s}$ spectrum near resonance (see Figure \ref{fig: comparing Is spectrum}b), the $I_{s}$ also becomes "wider". Meaning  that the detuning range between the maximal detunings $\Delta_{max}$ on both sides of the spectrum become larger. Furthermore, this effect also applies on both maxima points near resonance. This effect of "spectrum broadening" can be seen in Figure \ref{fig: compare Is spectrums}. The effect of turning-on the coupling laser by changing its Rabi frequency from $0$ to $4.2\gamma_{eg}$ (Figure \ref{fig: compare Is spectrums}a to \ref{fig: compare Is spectrums}b respectively) essentially introduces a new fluctuations near the resonance with a peak value similar to the peak value of superflash at maximal detuning $|\Delta_{max}|$. Visually, we can also see that the spectrum structure in the range of  $\Delta \approx -5\gamma_{eg}$ up to $\Delta \approx 5\gamma_{eg}$ in Figure \ref{fig: compare Is spectrums}a seems to be "replicated" into 2 "copies". Each copy on the red-detuned and blue-detuned side of the spectrum as can be seen starting from Figure \ref{fig: compare Is spectrums}b.

\begin{figure}[h!]
    \centering
    \includegraphics[scale = 0.8]{Figures/comparisonSpectral.PNG}
    \caption{$I_{s}$ spectrum for 4 different values of coupling laser's Rabi frequency $\Omega_{c}$. All of it have same OD of 62. Same probe laser parameters. a) $\Omega_{c} = 0$. b) $\Omega_{c} = 4.2\gamma_{eg}$. c) $\Omega_{c} = 8.4\gamma_{eg}$. d) $\Omega_{c} = 16.8\gamma_{eg}$. Note that $\gamma_{eg} = \frac{\Gamma_{e}}{2} = 1.885 \cdot 10^{7} \frac{rad}{sec}$}
    \label{fig: compare Is spectrums}
\end{figure}

As the coupling laser's Rabi frequency $\Omega_{c}$ are increased, the whole spectrum "widens" while also making the new fluctuations near resonance (introduced in Figure \ref{fig: compare Is spectrums}a) to become clearer and less convoluted. Figure \ref{fig: compare Is spectrums} showcase visually and more clearly the effect of increasing value of the maximal detuning $\Delta_{max}$ as the coupling laser's Rabi frequency $\Omega_{c}$ increases that is noted in the detuning value from Figure \ref{fig: corresponding temporal profile 1} up till Figure \ref{fig: corresponding temporal profile 3}. For the coupling laser's Rabi frequency of $0$, $4.2\gamma_{eg}$ and $8.4\gamma_{eg}$, the maximal detuning $\Delta_{max}$ increase from $10.8488\gamma_{eg}$ to $11.2467\gamma_{eg}$ then to $12.3077\gamma_{eg}$ while for the near-resonance maxima, the detuning value change from none (because there's no near-resonance maxima formed for $\Omega_{c} = 0$) to $0.3714\gamma_{eg}$ then to $1.2997\gamma_{eg}$.

\begin{figure}[h!]
    \centering
    \includegraphics[scale = 0.8]{IsPhaseTrans.png}
    \caption{4 subfigures each with different coupling laser's Rabi frequency $\Omega_{c}$. For each subfigure, the $I_{s}$ spectrum are plotted separately while the steady-state transmission $I_{ss}$ (solid green line) and the phase shift $\phi$ (solid blue line) are plotted together. a) $\Omega_{c} = 0$. b) $\Omega_{c} = 4.2\gamma_{eg}$. c) $\Omega_{c} = 8.4\gamma_{eg}$. d) $\Omega_{c} = 16.8\gamma_{eg}$. Note that $\gamma_{eg} = 1.8850 \cdot 10^{7} \frac{rad}{sec}$.}
    \label{fig: Is compared to transmission & phase shift}
\end{figure}

Both effects on the $I_{s}$ spectrum as the coupling laser's Rabi frequency is increased (the formation of fluctuations near-resonance and the "widening" of the spectrum) can be understood from the behaviour of the phase shift of the transmitted field $\phi$ and the steady-state transmission spectrum $I_{ss}$ as shown in Figure \ref{fig: Is compared to transmission & phase shift}.\\

\textit{Analysis & Explicit Expressions}
%analysis & explicit expressions

To summarize, the effect of EIT by switching-on the coupling laser is to create a fluctuation near resonance in the $I_{s}$ spectrum that corresponds to a change from an anomalous dispersion ($\dv{\phi(\Delta)}{\Delta} < 1$) to a normal dispersion ($\dv{\phi(\Delta)}{\Delta} > 1$) condition around resonance in the phase shift $\phi$ plot. In addition to that, It will also create a transparency window near-resonance (with a certain FWHM) that enables the value of the steady-state transmission $I_{ss}$ to approach unity. On the other hand, by increasing the coupling laser's Rabi frequency $\Omega_{c}$, the plot will "widen" causing 2 main effect. The 1st effect is related to the $I_{s}$ spectrum and the 2nd effect is related to the steady-state transmission spectrum $I_{ss}$.\\

Firstly, for the $I_{ss}$ spectrum. When the coupling laser are switched on going from the switched-off case ($\Omega_{c} = 0$) to the switched-on case ($\Omega_{c} = 4.2\gamma_{eg}$), it creates a new fluctuation near-resonance that produce a symmetrical structure which have 2 peaks that have $I_{s}$ values very close to the off-resonance global maximum $I_{s}$ values at maximal detuning $\Delta_{max}$. Therefore, this allows us to create a transmitted pulse with near-maximum $I_{s}$ value (The $I_{s}$ spectrum in Figure \ref{fig: Is compared to transmission & phase shift} are limited to around $3I_{0}$ because of its OD with value of 62. As discussed in \cite{Kwong2017}, higher $I_{s}$ value can be obtained by using a medium with a very large OD) at multiple detuning $|\Delta|$ value instead of only one. Furthermore, because the detuning value at the maximum $I_{s}$ shifts as the coupling laser's Rabi frequency $\Omega_{c}$ is changed, $\Omega_{c}$ can act as a "selector" to select the desired detuning that outputs a maximum $I_{s}$ values. So now, instead of a fixed detuning, we will have a convenient option to shift around the detuning only by manipulating the strength of the coupling laser's electric field (related to its Rabi frequency $\Omega_{c}$).\\

Finally, for the steady-state transmission $I_{ss}$ spectrum, as $\Omega_{c}$ increases, the Full-Width Half-Max (FWHM) value of the spectrum will also increase. As shown in Figure \ref{fig: Is compared to transmission & phase shift}, this would allow for more of the near-resonant frequency components in the incident field to be transmitted through the medium. This is actually a very well-known result and have been studied thoroughly in many other research papers. Refer \cite{} for more details on the relationship between the FWHM of the transmission spectrum and the coupling laser's Rabi frequency $\Omega_{c}$.
