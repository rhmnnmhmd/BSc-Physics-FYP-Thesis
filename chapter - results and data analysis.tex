\chapter{Results and Data Analysis}\label{results and analysis}
As this thesis are focused on coherent light-matter interactions specifically the coherent superflash phenomena, the results and analysis that will be presented in this chapter will be crucially related to that phenomena.

The results and analysis presented in this chapter heavily focused on the extension of coherent superflash research in a new kind 3-level EIT medium. The idea of extending the research of coherent superflash into 3-level EIT medium mainly came from these 2 research papers one published in Optic Letters \cite{jeong2010slow} and the other in Physical Review Letters \cite{Kwong2014}. From the former publication, the main idea incorporated into this project is the explicit expression of the electric susceptibility for 3-level EIT medium. From the later publication, the main idea implemented in this project is the method of studying superflash behaviour through the plot of forward-scattered intensity $I_{s}$ against incident probe detuning $\Delta$ basically its spectrum.

So, in this chapter, it will be divided to 4 sections. The 1st section in this chapter dedicate to temporal profiles generated using the programming code that was developed throughout this project. Its purpose is to show the validity of the code and how it's able to reproduce temporal profiles in many previous research papers. Next, for the 2nd section in this chapter, the focus will shift to the forward-scattered intensity spectrum (forward-scattered intensity $I_{s}$ vs incident probe laser detuning $\Delta$). The difference between the forward-scattered intensity spectrum that was produced from the usual 2-level medium and the 3-level EIT-enabled medium will be emphasized here. We'll see a new behaviour around the resonances region that can be counted as a new finding from this project has never been any studies on the forward-scattered intensity spectrum for 3-level EIT-enabled medium. Going further, in the 3rd section, the newly-discovered pattern of the forward-scattered spectrum for 3-level EIT medium will be justified by comparing its peak values (near-resonance and off-resonance) with their corresponding temporal profile. Finally, in the 4th section, the effect of varying the coupling laser's Rabi frequency on the forward-scattered intensity spectrum will be shown through the different output plots. We then propose an explanation for this effect by relating it to the detuning of the incident probe field $\Delta$ and the phase shift of the transmitted field $\Delta\phi$.


\section{Reproduction of Temporal Profiles}
From Figure \ref{fig: comparing temporal profile 1}, we can see that the Python code developed through this project were able to simulate the transmission intensity temporal profile of \cite{jeong2010slow} successfully. This serve to show that the simulation code developed from this project (More details discussed in the Methodology chapter) are valid and reliable enough to be utilized as the main tool for researching the coherent superflash phenomena.

\begin{figure}
    \centering
    \includegraphics[width = 1\textwidth]{Figures/default2.png}
    \caption[Comparison of Transmitted Intensity Temporal Profile From \protect\cite{jeong2010slow} and This Project's Simulation]{Comparison of the transmitted intensity temporal profile for square light pulse incident on 3-level EIT-enabled medium. a) Plot of the temporal profile using hybrid analysis of \protect\cite{jeong2010slow} (solid blue curve) and using FFT (black dashed curve). Figure courtesy of \protect\cite{jeong2010slow}. b) Temporal profile for the same parameters generated from Python code developed in this project.}
    \label{fig: comparing temporal profile 1}
\end{figure}

The code for simulating the transmission temporal profile are really crucial for this project and must be sufficiently accurate because all further results/plots after this heavily depends on the temporal profile code. The next result discusses the forward-scattered intensity spectrum $I_{s}$. The plots displayed in that section (values of $I_{s}$) are extracted at the abrupt extinction point in the temporal profile varied over different values of incident probe detuning $\Delta$. Therefore, if Figure \ref{fig: comparing temporal profile 1}a can't be properly reproduced with correct values by the code, all further outputs would be pointless.

Analysis of the plot in Figure \ref{fig: comparing temporal profile 1}a have been done in details in \cite{jeong2010slow}. In that paper, they only considered the on-resonance case (incident probe laser detuning $\Delta = 0$) while in \cite{Kwong2014}, the off-resonance is considered. In terms of superflash studies, the (peak) value of transmitted intensity at the abrupt-extinction region is exactly equivalent to the forward-scattered intensity $I_{s} = |E_{s}|^{2}$. The mechanism for how the value of this forward-scattered intensity $I_{s}$ varies with probe detuning $\Delta$ are discussed in detail in \cite{Chalony2011, Kwong2014, Kwong2017}. In short, the value of forward-scattered intensity $I_{s}$ depends on the phase of the forward-scattered field $\phi_{s}$ which also depends on the detuning of the incident probe laser $\Delta$. Therefore, certain detuning of the incident probe laser can produce forward-scattered field with a phase such that it will have a magnitude of more than 1 (basically superflash phenomena) at the abrupt-extinction point.

The Figure \ref{fig: theta_s} that was taken from \cite{Kwong2014} summarized this idea concisely where we can notice that the magnitude of the forward-scattered field are not necessarily bounded to the magnitude of the incident electric field. In addition to that, the interference between the incident field and forward-scattered field doesn't have anything to do with the values of the forward-scattered intensity at the abrupt-extinction point. This is true because during that point in time, the incident probe laser has been turned off and the forward-scattered field are actually radiated by the "residual" dipoles in the medium that doesn't instantaneously decay to $0$.

This is also related to the cooperative emission effect that was mentioned in \cite{Kwong2014, Kwong2015, Kwong2017, Araujo2016} that is prevalent in optically thick medium. Its main effect is to reduce the decay time (inverse of atomic excited-state decay rate $\Gamma$) of the superflash itself by a certain different factors depending on the modulation of the incident field. The specific expression for these factors can be found in \cite{Kwong2017} and they also mentioned that there are a common factor even for different type of modulation that doesn't depend on the temperature of the medium or even the detuning of the probe field. The common factor is $\frac{2}{b_{0}(0)}$ where $b_{0}(0)$ are the resonant optical depth of the medium at zero-temperature which is the main parameter that governs this cooperative effect.

To demonstrate the dependence of the forward-scattered intensity $I_{s}$ to the incident probe laser detuning $\Delta$, some temporal profiles are shown in Figure \ref{fig: comparing temporal profile 2} but instead of zero-detuning case as produced by \cite{jeong2010slow} as shown in Figure \ref{fig: comparing temporal profile 1}a, the temporal profiles are for the case of non-zero detuning. Both are generated from the simulation code that is developed in this project.

\begin{figure}
    \centering
    \includegraphics[width = 1\textwidth]{Figures/compareTemporal3.png}
    \caption[Plots of Transmitted Intensity Temporal Profile for Different Detunings]{Transmission intensity temporal profile of an incident square pulse (pulse duration of $2\cdot10^{-6}$ sec) on cold, dilute atomic gas of Rb-85 with optical depth (OD) of 62 and coupling laser's Rabi frequency of $4.2\gamma_{eg}$ where $\gamma_{eg}$ is half of the natural linewidth $\gamma_{eg} = \frac{\Gamma_{e}}{2} = 1.8850 \cdot 10^{7} \frac{rad}{sec}$. Each only have different probe detuning $\Delta$. a) $\Delta = 5.6764\gamma_{eg}$. b) $\Delta = 0.3714\gamma_{eg}$. c) $\Delta = 11.2467\gamma_{eg}$.}
    \label{fig: comparing temporal profile 2}
\end{figure}

We could get both the desired coherent superflash (Figure \ref{fig: comparing temporal profile 2}b and \ref{fig: comparing temporal profile 2}c) and the coherent antiflash also mentioned in \cite{Kwong2017} (Forward-scattered intensity lower than incident intensity. Figure \ref{fig: comparing temporal profile 2}a). In addition to that, the coherent superflash produced in Figure \ref{fig: comparing temporal profile 2}b and \ref{fig: comparing temporal profile 2}c are significantly higher than the one obtained in \cite{jeong2010slow} with a value of about $3I_{0}$. These values are approaching the upper limit value of $4I_{0}$ according to the discussion in \cite{Kwong2014}.

The resulting coherent superflash in both Figure \ref{fig: comparing temporal profile 2}b and \ref{fig: comparing temporal profile 2}c have peak value similar to the result obtained in Figure 1c of \cite{Kwong2014}. They obtained a peak value of about $3.1 I_{0}$. Since Figure 1c of \cite{Kwong2014} are obtained just using a 2-level medium, this doesn't really show any advantages of using 3-level EIT medium over the former. This only supports the idea that the peak value of a coherent superflash can be optimized to be closer to the upper limit value of $4I_{0}$ by detuning the probe laser by a specific amount. As shown in \cite{Kwong2014}, for a 2-level medium, the value of the coherent superflash can be made closer to $4I_{0}$ by using a medium with larger optical depth $OD$ and detuning the incident probe laser to the maximal detuning $\Delta_{max}$. For a medium with large OD, they acquired an expression for the maximal detuning;

\begin{equation}
    \abs{\Delta_{max}} \approx \frac{b_{0}(0) \Gamma}{4\pi} 
\end{equation}

\newpage

Where;

\begin{itemize}
    \item $b_{0}(0) = \frac{b_{\Bar{v}}(0)}{g(\frac{kv}{\Gamma})}$ is the resonant OD at zero-temperature. 
    \item $b_{\Bar{v}}(0)$ is the resonant OD at non-zero temperature.
    \item $g(\frac{kv}{\Gamma})$ is a real function related to the Complementary Error Function $Erfc(x)$ \cite{abramowitz1965ia}.
\end{itemize}

In addition to that, they also showed that at this maximal detuning $\Delta_{max}$, the phase shift $\Delta\phi$ of the transmitted field $E_{t}(t, z)$ with respect to the incident field $E_{0}(t, z)$ will always be $\approx \pi$.

So, in this project, similar analysis have been done for a 3-level EIT medium to acquire an explicit expression for the maximal detuning $\Delta_{max}$. The explicit expression that was obtained are discussed in more details in Section \ref{OmegaCDependence}. This explicit expression of the maximal detuning for 3-level EIT system would be really useful. The research community can use this expression to obtain the appropriate incident probe detuning value that will maximize the superflash value in their setup. Most importantly, this maximal detuning value $\Delta_{max}$ is tunable through the incident coupling laser's Rabi frequency $\Omega_{c}$.


\section{Forward-Scattered Intensity Spectrum for 3-Level EIT Medium}
In Figure \ref{fig: comparing Is spectrum}. The $I_{s}$ spectrum for a 3-level medium was compared. The parameters and important equations (e.g. electric susceptibility $\chi$) are based on \cite{jeong2010slow}. This idea of comparing the $I_{s}$ spectrum are inspired by Figure 1d of \cite{Kwong2014} where they had plotted the $I_{s}$ spectrum for their 2-level medium while in here it is done for a 3-level EIT medium. In Figure \ref{fig: comparing Is spectrum}a, even though it is a 3-level EIT medium, the $I_{s}$ spectrum are similar to a 2-level medium's spectrum for example the ones plotted in Figure 1d of \cite{Kwong2014} which is a 2-level medium. This is because the coupling laser's Rabi frequency $\Omega_{c}$ is set to $0$. The coupling laser's Rabi frequency $\Omega_{c}$ basically acts as a "switch" for the EIT effect. When it is "switched-off", the coupling between the 3rd new coupling energy level $\ket{c}$ with the excited-state energy level $\ket{e}$ doesn't occur therefore no EIT effect present. A nice schematic of this said energy levels configuration from \cite{Jeong2009} can be seen in Figure \ref{fig: 3LevelAtom}.

\begin{figure}
    \centering
    \includegraphics[width = 1\textwidth]{Figures/compareSpectral.png}
    \caption[Plot of $I_{s}$ Spectrum For 3-Level EIT-Disabled and EIT-Enabled Medium]{The forward-scattered intensity detuning spectrum ($I_{s}$ spectrum from now on) for the same initial parameters but a) Without EIT (Coupling laser's Rabi frequency $\Omega_{c} = 0$) while b) With EIT (Coupling laser's Rabi frequency $\Omega_{c} = 4.2\gamma_{eg}$ where $\gamma_{eg} = 1.885 \cdot 10^{7} \frac{rad}{sec}$).}
    \label{fig: comparing Is spectrum}
\end{figure}

\begin{figure}
    \centering
    \includegraphics[width = 0.5\textwidth]{Figures/3LevelAtom.png}
    \caption[Schematic of The Energy Levels Involved in A 3-Level EIT Medium]{Note that the different notation used here for the energy levels. $\ket{1}$ is equivalent to $\ket{g}$. $\ket{2}$ is equivalent to $\ket{c}$. $\ket{3}$ is equivalent to $\ket{e}$. Figure courtesy of \protect\cite{Jeong2009}.}
    \label{fig: 3LevelAtom}
\end{figure}

While for Figure \ref{fig: comparing Is spectrum}b, the coupling laser's Rabi frequency $\Omega_{c}$ was set to $4.2\gamma_{eg}$ where $\gamma_{eg} = \frac{\Gamma_{e}}{2} = 1.885 \cdot 10^{7} \frac{rad}{sec}$. All the other parameters for example the optical depth (OD = 62) are according to \cite{jeong2010slow}. As observed in Figure \ref{fig: comparing Is spectrum}b, a nearly-symmetrical new spectral feature started to form near the resonance while the other parts of the spectrum remained unchanged. In addition to that, the peak value of this new spectral feature is really close to the peak value of the superflash at the wing of this $I_{s}$ spectrum (this peak value at the wing occurs at maximal detuning $\Delta_{max}$ as discussed for 2-level medium in \cite{Kwong2017}). This is a new and very interesting result from this project that had never been acquired before.

Through combining some crucial ideas from \cite{Kwong2014} and \cite{jeong2010slow}, we were able to obtain the new $I_{s}$ spectrum for a 3-level EIT medium. The new feature of the $I_{s}$ spectrum shown here are definitely interesting because we obtained an alternate method to produce a new superflash value (near-resonance) that is close to the peak superflash value (at maximal detuning $\Delta_{max}$). The previous method (in 2-level medium) as shown in \cite{Kwong2017} was by using a medium with larger OD because as can be seen in Fig 5.10 and 5.11 of \cite{Kwong2017}, medium with larger OD will have multiple points with superflash values close to the global maximum value. So, by using a 3-level EIT medium, a new superflash points (with values close to the global maximum superflash value) can be created by controlling the coupling laser's Rabi frequency $\Omega_{c}$ instead of manipulating the OD of the target medium.


\section{Corresponding Temporal Profile for The Maximum Points in The 3-Level EIT Medium's Forward-Scattered Intensity Spectrum}
In this section, it is dedicated to show the validity of the $I_{s}$ spectrum that were obtained for the 3-level EIT-enabled medium using the code by comparing the peak values in the $I_{s}$ spectrum with the peak value of transmitted intensity (which occurs at the extinction point and are equivalent to $I_{s}$) in their corresponding temporal profile. There will be 3 cases in this section. Each cases are for different values of coupling laser's Rabi frequency $\Omega_{c}$. The first, second and the third case will have incident coupling laser's Rabi frequency $\Omega_{c}$ of $0$, $4.2\gamma_{eg}$ and $8.4\gamma_{eg}$ respectively.

Firstly, in Figure \ref{fig: corresponding temporal profile 1}, the $I_{s}$ spectrum are for the $\Omega_{c} = 0$ case (which is equivalent to a 2-level medium). All other parameters e.g. OD and medium atomic species are according to \cite{jeong2010slow}. The detuning points that are chosen are the local maxima near resonance ($\Delta = 3.1565\gamma_{eg}$) and the global maxima at the wing of the spectrum ($\Delta = 10.8488\gamma_{eg}$). As can be clearly seen, the value of $I_{s}$ in both temporal profiles agree closely with the value of $I_{s}$ in the spectrum at their respective probe detuning. Therefore, for the 3-level EIT-disabled case here, $I_{s}$ value in the spectrum are able to represent accurately and excellently in the temporal profile at the corresponding detuning.

\begin{figure}
    \centering
    \includegraphics[width = 1\textwidth]{Figures/spectralTemporal1.png}
    \caption[Plots of Transmitted Intensity Temporal Profile At Their Corresponding Detuning in The $I_{s}$ Spectrum For $\Omega_{c} = 0$]{The corresponding temporal profiles at local maxima point in the $I_{s}$ spectrum b) near resonance and c) off-resonance. a) $I_{s}$ spectrum for 3-level EIT-disabled medium ($\Omega_{c} = 0 \frac{rad}{sec}$). b)  Temporal profile at probe detuning $\Delta = 3.1565\gamma_{eg}$. c) Temporal profile at probe detuning $\Delta = 10.8488\gamma_{eg}$.}
    \label{fig: corresponding temporal profile 1}
\end{figure}

Next, in Figure \ref{fig: corresponding temporal profile 2}, the $I_{s}$ spectrum are for the $\Omega_{c} = 4.2\gamma_{eg}$ case. Making it a 3-level EIT medium. All other parameters are still the same as before. The detuning values that are chosen are the maxima point near resonance ($\Delta = 0.3714\gamma_{eg}$. The value are really close to the global maxima's value) and the global maxima point ($\Delta = 11.2467\gamma_{eg}$). As shown in the figure, the value of $I_{s}$ in both temporal profiles also agree closely with the value of $I_{s}$ in the spectrum at their respective probe detuning. At the first point, $I_{s}$ is about $2.7I_{0}$ while at the 2nd chosen point, it's about $3I_{0}$. So, for the case here, the $I_{s}$ value in the $I_{s}$ spectrum are also able to represent the $I_{s}$ value in the temporal profile at the corresponding detuning.

\begin{figure}
    \centering
     \includegraphics[width = 1\textwidth]{Figures/spectralTemporal2.PNG}
    \caption[Plots of Transmitted Intensity Temporal Profile At Their Corresponding Detuning in The $I_{s}$ Spectrum For $\Omega_{c} = 4.2\gamma_{eg}$]{The corresponding temporal profiles at local maxima point in the $I_{s}$ spectrum b) near resonance and c) off-resonance. a) $I_{s}$ spectrum for 3-level EIT-enabled medium ($\Omega_{c} = 4.2\gamma_{eg}$). b) Temporal profile at probe detuning $\Delta = 0.3714\gamma_{eg}$. c) Temporal profile at probe detuning $\Delta = 11.2467\gamma_{eg}$.}
    \label{fig: corresponding temporal profile 2}
\end{figure}

Finally, in Figure \ref{fig: corresponding temporal profile 3}, the $I_{s}$ spectrum are for the $\Omega_{c} = 8.4\gamma_{eg}$ case. Making it also a 3-level EIT medium like the 2nd case. All parameters e.g. OD and medium atomic species are unchanged and identical to the previous case. The detuning values are chosen for the maxima point near resonance ($\Delta = 1.2997\gamma_{eg}$. The value are also really close to the global maxima's value) and the global maxima point at the wing of the spectrum ($\Delta = 12.3077\gamma_{eg}$). As expected, the value of $I_{s}$ in both temporal profiles agree with the value of $I_{s}$ in the spectrum at their respective probe detunin  g. At both chosen points, the $I_{s}$ values are about $3I_{0}$. Hence, for the case here, the $I_{s}$ value in the $I_{s}$ spectrum are also able to represent the $I_{s}$ value in the temporal profile at the respective detuning.

\begin{figure}
    \centering
    \includegraphics[width = 0.93\textwidth]{Figures/spectralTemporal3.png}
    \caption[Plots of Transmitted Intensity Temporal Profile At Their Corresponding Detuning in The $I_{s}$ Spectrum For $\Omega_{c} = 8.4\gamma_{eg}$]{The corresponding temporal profiles at local maxima point in the $I_{s}$ spectrum b) near resonance and c) off-resonance. a) $I_{s}$ spectrum for 3-level EIT-enabled medium ($\Omega_{c} = 8.4\gamma_{eg}$). b) Temporal profile at probe detuning $\Delta = 1.2997\gamma_{eg}$. c) Temporal profile at probe detuning $\Delta = 12.3077\gamma_{eg}$.}
    \label{fig: corresponding temporal profile 3}
\end{figure}

In summary, for every cases presented here, each of the $I_{s}$ spectrum have the same values with the $I_{s}$ values in the corresponding temporal profiles at the selected detuning points. It can also be noted that for both the near-resonance maxima points and the global maxima points at the wing of the spectrum, as the coupling laser's Rabi frequency $\Omega_{c}$ is increased, their respective detuning value also increase. So, it is hypothesized that this detuning $\Delta$ are directly proportional to the coupling laser's Rabi frequency $\Omega_{c}$. This hypothesis will be discussed in more details in the next section with a proposed relevant equations.


\section{Relation between The Forward-Scattered Intensity Spectrum with The Coupling Laser's Rabi Frequency}\label{OmegaCDependence}
As mentioned in the previous section, it is observed that, when the coupling laser's Rabi frequency are switched-on and as its value are increased, aside from the production of a new spectral feature in the $I_{s}$ spectrum near resonance (see Figure \ref{fig: comparing Is spectrum}b), the $I_{s}$ spectrum also becomes "wider". Meaning  that the detuning range between the maximal detunings $\Delta_{max}$ on both sides of the spectrum become larger. Furthermore, this effect also applies on both local maxima points near resonance. This effect of "spectrum broadening" can be seen in Figure \ref{fig: Is compared to transmission & phase shift}. The effect of turning-on the coupling laser by changing its Rabi frequency from $0$ to $4.2\gamma_{eg}$ (Figure \ref{fig: Is compared to transmission & phase shift}a to \ref{fig: Is compared to transmission & phase shift}b respectively) essentially introduces a new spectral feature near the resonance with a peak value that is quite close to the peak value of superflash at the global maxima points (at the wing of the spectrum having maximal detuning $\Delta_{max}$). Visually, we can also see that the spectrum structure in the range of  $\Delta \approx -5\gamma_{eg}$ up to $\Delta \approx 5\gamma_{eg}$ in Figure \ref{fig: Is compared to transmission & phase shift}a seems to be "replicated" into 2 "copies". Each copy on the red-detuned and blue-detuned side of the spectrum as can be seen starting from Figure \ref{fig: Is compared to transmission & phase shift}b.

\begin{figure}
    \centering
    \includegraphics[width = 1\textwidth]{Figures/compareSpectral2.PNG}
    \caption[Plot of $I_{s}$ Spectrum, Steady-State Transmission Intensity and The Phase Shift]{4 subfigures each with different coupling laser's Rabi frequency $\Omega_{c}$. For each subfigure, the $I_{s}$ spectrum are plotted separately while the steady-state transmission $I_{ss}$ (solid green line) and the phase shift $\phi$ (solid blue line) are plotted together. a) $\Omega_{c} = 0$. b) $\Omega_{c} = 4.2\gamma_{eg}$. c) $\Omega_{c} = 8.4\gamma_{eg}$. d) $\Omega_{c} = 16.8\gamma_{eg}$. Note that $\gamma_{eg} = 1.8850 \cdot 10^{7} \frac{rad}{sec}$.}
    \label{fig: Is compared to transmission & phase shift}
\end{figure}

As the coupling laser's Rabi frequency $\Omega_{c}$ are increased, the whole spectrum "widens" while also making the new near-resonance spectral feature (introduced in Figure \ref{fig: Is compared to transmission & phase shift}b) to become more resolved. From Figure \ref{fig: Is compared to transmission & phase shift}, we can visually see how the global maxima point shifts further away from the resonance. This also correspond to an increase in the value of the maximal detuning $\Delta_{max}$.

In terms of numerical detuning values, the maximal detuning $\Delta_{max}$ increase from $10.8488\gamma_{eg}$ to $11.2467\gamma_{eg}$ then to $12.3077\gamma_{eg}$ and finally $15.3581\gamma_{eg}$. For the near-resonance maxima, the detuning value change from none (because there's no near-resonance maxima formed for $\Omega_{c} = 0$) to $0.3714\gamma_{eg}$ then to $1.2997\gamma_{eg}$ and finally to $4.4828\gamma_{eg}$.

Both effects (the formation of the new spectral feature near-resonance and the widening of the spectrum) on the $I_{s}$ spectrum as the coupling laser's Rabi frequency is increased  can be understood from the behaviour of the phase shift $\phi$ of the transmitted field $E_{t}$ with respect to the incident field $E_{0}$ and the steady-state transmission spectrum $I_{ss}$ as shown in Figure \ref{fig: Is compared to transmission & phase shift}.

Firstly, for the formation of the new spectral feature near resonance. From Figure \ref{fig: Is compared to transmission & phase shift}, we can see that this new spectral feature have about the same spectral width when compared to the spectral width of the transparency window (created from EIT effect) in the steady-state transmission spectrum. In addition to that, when we look at the phase shift's spectrum, we notice that this new spectral feature is also in the slow-light region where the group velocity $v_{g}$ of the steady-state transmission is slower than the vacuum speed of light $c$ according to Equation \ref{eqn: groupVelocity}. We can see how by changing the coupling laser's Rabi frequency $\Omega_{c}$ from $0$ to a non-zero finite value (from Figure \ref{fig: Is compared to transmission & phase shift}a to Figure \ref{fig: Is compared to transmission & phase shift}b), it immediately changes the condition near resonance (where the transparency window and the new spectral feature started to form) from anomalous dispersion ($\dv{\phi(\Delta)}{\Delta} < 0$, $v_{g} > c$) to a normal dispersion ($\dv{\phi(\Delta)}{\Delta} > 0$, $v_{g} < c$). This normal dispersion condition are unchanged over the transparency window. This idea regarding transparency window and slow-light have been researched extensively in many publications \cite{jeong2010slow, MacKe2009, Jeong2009, MacKe2009, Braje2004}.

Finally, for the widening of the $I_{s}$ spectrum. From Figure \ref{fig: Is compared to transmission & phase shift}, we can see visually how the global maxima points (with maximal detuning $\Delta_{max}$) and the new maxima points near resonance (from the newly-formed spectral feature) shifts away from the resonance as the coupling laser's Rabi frequency $\Omega_{c}$ is increased. Previously we have discussed the numerical values of the maximal detuning $\Delta_{max}$ and the detuning value for the maxima point near resonance (which we will call "near-resonance maximal detuning" $\Delta_{nmax}$ from now on). Now, we will show why both these type of maxima points occurs at their respective detuning. From \cite{Kwong2014, Kwong2017}, we know that one of the ways to improve the superflash value closer to the limit $4I_{0}$ is by using a medium with higher $OD$ because this will enable an incident probe laser detuning value that can achieve that limit (see Figure \ref{fig: I_s/I_0}). Then, based on Figure \ref{fig: theta_s}, this said probe detuning value corresponds to a forward-scattered field $E_{s}$ with a phase shift $\abs{\Delta\phi} \approx \pi$ and have forward-scattered electric field $E_{s}$ with a magnitude closer to $2$ as the $OD$ is increased. Basically, the same condition also applies to the 3-level EIT medium studied in this project. This can be shown by the numerical value of the phase shift $\Delta\phi$  (in Figure \ref{fig: omegaPhase}) when the probe detuning is equal to the maximal $\Delta_{max}$ and the near-resonance maximal $\Delta_{nmax}$ detuning.

From now on, we will discuss the values on the positive side of the detuning spectrum for simplicity and clarity (The general behaviour is the same even on the negative side of the spectrum). Referring to figure \ref{fig: omegaPhase}a when the EIT effect is turned off ($\Omega_{c} = 0$), at the maximal detuning ($\Delta_{max} = 10.8488\gamma_{eg}$), we have the phase shift $\Delta\phi = -0.9020\pi$. Next, for the case where EIT effect is turned on in Figure \ref{fig: omegaPhase}b, \ref{fig: omegaPhase}c and \ref{fig: omegaPhase}d with coupling laser's Rabi frequency $\Omega_{c} = 4.2\gamma_{eg}$, $\Omega_{c} = 8.4\gamma_{eg}$ and $\Omega_{c} = 16.8\gamma_{eg}$ respectively. For each cases, We will have two values of phase shifts $\Delta\phi$. Each of the phase shift values are for the maxima point in the $I_{s}$ spectrum that is near resonance (with $\Delta_{nmax}$) and at the wing of the spectrum (with $\Delta_{max}$). For $\Omega_{c} = 4.2\gamma_{eg}$ case, we have $\Delta\phi = 0.8492\pi$ and $\Delta\phi = -0.9015\pi$.  

\begin{figure}
    \centering
    \includegraphics[width = 1\textwidth]{Figures/omegePhase.PNG}
    \caption[Plot of $I_{s}$ Spectrum Compared With The Phase Shift]{Plot of $I_{s}$ spectrum compared to their corresponding phase shift spectrum for a) $\Omega_{c} = 0$, b) $\Omega_{c} = 4.2\gamma_{eg}$, c) $\Omega_{c} = 8.4\gamma_{eg}$ and d) $\Omega_{c} = 16.8\gamma_{eg}$. The circle signifies the point where the detuning $\Delta$ is equal to the maximal $\Delta_{max}$ and near-resonance maximal $\Delta_{nmax}$ detuning.}
    \label{fig: omegaPhase}
\end{figure}

Next,  for $\Omega_{c} = 8.4\gamma_{eg}$ case, we have $\Delta\phi = 0.7982\pi$ and $\Delta\phi = -0.8999\pi$. For the last case where $\Omega_{c} = 16.8\gamma_{eg}$, we have $\Delta\phi = 0.8694\pi$ and $\Delta\phi = -0.9089\pi$. As we can clearly observe, all of the phase shift values have an absolute value that is really close $\pi$. The largest percentage error is about $20.18\%$ for the $\Omega_{c} = 8.4\gamma_{eg}$ case at the detuning $\Delta_{nmax} = 1.2977\gamma_{eg}$ while the smallest percentage error is $9.11\%$ for the $\Omega_{c} = 16.8\gamma_{eg}$ case at the detuning $\Delta_{max} = 15.3581\gamma_{eg}$.

Therefore, it can be concluded that the maxima point in the $I_{s}$ spectrum occurs at their respective detuning such that the absolute value of the phase shift of the field have an absolute value $\approx \pi$. Furthermore, because the maximal and the near-resonance maximal detuning is directly proportional to the coupling laser's Rabi frequency $\Omega_{c}$, it can act as a "selector" or "controller" to select the desired detuning that outputs a maximum $I_{s}$ values. So now, instead of a fixed detuning, we will have a convenient option to shift around the detuning only by manipulating the strength of the coupling laser's electric field (related to its Rabi frequency $\Omega_{c}$).
