%!TEX root=thesis.tex
%This is the draft abstract. More soon, I promise!
%An abstract must not exceed 500 words, typed in a single paragraph with double- spacing, and written in Bahasa Malaysia and English language. A maximum of five (5) keywords should also be listed below the abstract.

Coherent superflash is one of the coherent transient phenomena that can be observed when an incident probe laser beam that propagates through a medium is abruptly turned off either through amplitude or phase modulation. To study more on the behavior of coherent superflash, a numerical simulation of the transmitted signal has been devised in the Python programming language. This language utilizes the Fast Fourier Transform (FFT) package to obtain the time-evolution of the transmitted signal $E_{t}(t, z)$ by calculating the Inverse Fast Fourier Transform (IFFT) of the transmitted signal's spectrum $E_{t}(\omega, z)$. The objective of this project is to extend the knowledge on coherent superflash regarding its characteristic parameters e.g. forward-scattered intensity $I_{s}$ to another new type of medium that is the 3-level Electromagnetically Induced Transparency (EIT) medium instead of the usual 2-level medium as discussed in \cite{Kwong2014}. The important result discovered from this project is the detuning tunability of the superflash that cannot be obtained in 2-level medium. Concisely, it can be stated that the result produced here showed that when the medium is changed from 2-level medium to a 3-level EIT medium, unique spectral features are observed near medium resonance contrary to the 2-level case where it is monotonous without any interesting feature.

%\textbf{Keywords: } Superflash, 3-Level EIT Medium, Forward-Scattered Intensity.
