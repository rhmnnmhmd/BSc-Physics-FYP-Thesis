%!TEX root=thesis.tex
%This is the draft abstract. More soon, I promise!
%An abstract must not exceed 500 words, typed in a single paragraph with double- spacing, and written in Bahasa Malaysia and English language. A maximum of five (5) keywords should also be listed below the abstract.

Coherent superflash is one of the coherent transient phenomena that can be observed when a probe laser beam that is incident on a medium is abruptly switched off either through amplitude or phase modulation of the incident signal. To study more on the behavior of coherent superflash, a numerical simulation of the transmitted signal has been devised in the Python programming language utilizing the Fast Fourier Transform (FFT) package to obtain the time-evolution of the transmitted signal $E(t, z)$ by calculating the Inverse Fast Fourier Transform (IFFT) of the transmitted signal's spectrum $E(\omega, z)$. The objective of this project is to extend the knowledge on coherent superflash regarding its characteristic parameters e.g. forward-scattered intensity $I_{s}$ to another new type of medium (3-level EIT medium) instead of the usual 2-level medium as been done in \cite{Kwong2014}. The important result discovered from this project is how the forward-scattered field's detuning spectrum for 3-level EIT medium that was produced here differs from Figure 1)d) that is produced in \cite{Kwong2014}. Concisely, it can be stated that the result produced here showed that when the medium is changed from 2-level medium to a 3-level EIT medium, fluctuations will start to be present in the region around the 0 detuning contrary to the 2-level version where there is very little fluctuation. In conclusion, the numerical simulation code from this project will be useful to the students or even researchers for studying both the temporal behaviour of the transmitted intensity and spectral behaviour of forward-scattered intensity.

\textbf{Keywords: } Superflash, 3-Level EIT Medium, Forward-Scattered Intensity.
